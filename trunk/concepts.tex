\section{The \eu Planner }
\label{sec:basics}

\paragraph {A brief note on \eu history and notation} The origin of
what we will call \eu in this chapter, is derived from the
\texttt{HSTS} planner \cite{mus94} used on a number of ground-based
demonstrations for space telescope scheduling as noted in Section
\ref{sec:rabeyond}. \texttt{HSTS} was also the \texttt{LISP}-based
planner that was flown as part of the Remote Agent \texttt{RAX}
experiment. Concurrent to \texttt{RAX} deployment, a C++ version of
the planner was implemented \kcomment{at NASA Ames. This} went on to
be used for the \texttt{MAPGEN} \cite{bresina05} system and continues
to be used to date, in the mission-critical uplink process for the
command/control of the Mars Exploration Rovers (\texttt{MER})
mission. A vastly re-factored, higher performance open source version
\cite{europapso} was named \eut and is the subject of this chapter and
used by \rxe.  Note however, in using the name \eu, we refer to the
\emph{fundamental ideas as well as the planning paradigm} even if
there exists a specific implementation.

\subsection{Introduction}
\label{sec:euintro}

\eu uses a \emph{domain model} \kcomment{(see Fig. \ref{fig:planner})}
written in a declarative language, together with initial conditions
and goals (also in that specified language), to construct a set of
temporal relations that must be true at start time. These models
include assertions about the physics of the vehicle, i.e how it
responds to external stimulus and internally driven goals. By
propagating these relations forward using Simple Temporal Networks
\cite{dechter91} and applying goal constraints, \eu can select a set
of conditions that should be true in the future, where some of these
conditions will correspond to actions the agent must take. The planner
can backtrack and try another path during the search process if a goal
cannot be reached while being capable of discarding unachievable
goals.

Traditionally robot execution has relied on dispatching commands at
precise times. Such linear sequences of precisely timed commands give
no ability to adjust execution on the basis of sensory
information. Although some commands can issue tests on sensor
readings, these tests have the objective of verifying whether expected
execution conditions are occuring. If not, the state of the system is
declared off-nominal and execution of the sequence of commands is
interrupted. More recently, executives have been proposed and
implemented that significantly broaden the way robots can be commanded
\cite{mus98,alami:1998p820}. For example, the Remote Agent executive
interpreted a \textit{temporally flexible plan} which represents each
start time as a \kcomment{flexible timepoint} variable
\kcomment{backed by} an explicit network of bounded delay constraints
between such variables. \kcomment{These timepoints \cite{Dean88}
  represent temporal intervals signifying a change in state and are of
  the form $[lb,ub]$ (where $lb, ub \in N$) are temporal lower and
  upper bounds respectively. Instead of specifying a fixed integer,
  time is represented as an interval}. Fig. \ref{fig:flex-timelines}
shows an example of a flexible partial plan.

\begin{figure}[!htb]
\centering
\includegraphics[scale=0.4]{figs/flexible-timelines.pdf}
\caption{\small Tokens with flexible temporal intervals and parametric
  constraints between tokens. This example shows the triggering of a
  water sampler based on a feature threshold while the vehicle
  Yo-Yo's. The \texttt{Waypoint\_Yo-Yo} token has a flexible duration,
  start \& end times.}
\label{fig:flex-timelines}
\end{figure}

Unlike a traditional fixed time-tagged command sequences, such
flexible plans leave room for adaptation at execution time. When the
executive considers when to start a task, it propagates information
through the constraint network, computes a time bound for the
variable, selects an actual execution time within the bound, and
starts the task at that time. Temporally flexible plans therefore,
express a \textit{range of possible outcomes} of the robots
interaction with the environment within which the executive can elect
at run time, the most appropriate for actual execution. The fact that
constraints are explicitly represented ensures that through constraint
propagation the executive will respect global limits expressed in the
plan (e.g., don't start a task until a certain condition has been
satisfied but still satisfying some overall deadline). Such
flexibility is critical when dealing with dynamic and uncertain
conditions \kcomment{for robotic action in real-world environments},
where precise timing of an action might be indeterminate. Further, the
advantage of flexibility can be contrasted with the consequences of
the intrinsic inflexibility of traditional command sequences. Because
they are inflexible, sequences \kcomment{are brittle and therefore}
must necessarily be designed considering worst case scenarios.

\eu uses a \emph{state variable} representation
\cite{mus94,jonsson00,py10} to describe the evolution of state over
time. The instantiated history of such \kcomment{an} evolution over a
temporal horizon we call \kcomment{as a} \emph{timeline} and
\kcomment{one which represents} a single thread in the execution of a
concurrent system. At any given time each thread can execute a single
procedure\kcomment{, with} each timeline consisting of a sequence of
procedures which encapsulate and describe state evolution. We call
these instantiated atomic entities \emph{tokens}.  A token therefore
describes a procedure invocation, the state variables on which it can
occur, the parameter values of the procedure, and the
\kcomment{timepoints} defining the interval \kcomment{(see
  Fig. \ref{fig:flex-timelines})}. We allow encapsulation of
uncertainty within these tokens with a range of start and end times
and parameters, all of which are encoded as variables. A constraint
solver in turn manipulates these variables defined in an \eu domain
model. For example, consider a scientific need to take water samples
100 meters from an environmental hotspot while an AUV is performing a
Yo-Yo sawtooth pattern\footnote{\kcomment{Variability in the water
    column is along the vertical dimension. Since the Dorado platform
    can only move forward, the Yo-Yo pattern is the most efficient
    mechanism for studying water column properties.}}  in the
water-column. Two samples are needed if the feature's signal is above
a threshold; one otherwise. The token that is capturing the sensory
threshold has a parametric constraints to the token which
\kcomment{triggers} the requisite water sampler. In addition, the
start time of the water sampling procedure is highly dependant on the
variability of sub-sea currents and actual speed of the
vehicle. Therefore a number of values are possible for the start times
and duration of the sampling all of which are valid combinations for
desired outcomes.

\subsection{\eu Plan Representation}
\label{sec:europa:pr}

\eu follows the representation outlined by \texttt{CAIP} augmented
with elements that are necessary for an efficient implementation of
constraint-based planning. We briefly describe the main elements,
followed by a detailed description for each of these elements and how
together they constitute a rich representational paradigm for
inference in Sections \ref{sec:europa:modeling},
\ref{sec:europa:inference} and \ref{sec:europa:search}.

\begin{enumerate}

\item \textbf{Tokens} Temporally scoped entities that correspond to
  Intervals in \texttt{CAIP} that are used to represent state and
  actions in time. A\textbf{Token State model} is defined to support
  an efficient implementation of plan-space search algorithms.

\item \textbf{Domains, Variables and Constraint} are used to represent
  restrictions in the problem domain in terms of a \texttt{CP}
  formulation.

\item \textbf{Domain Rules} are used to describe internal and external
  relationships on and between tokens, as presecribed by
  \texttt{CAIP}.
  
\item \textbf{Objects} are used to represent problem domains in a more
  modular and scaleable way. \textbf{Timelines and Resources} are
  built in implementations of object types that appear frequently in
  real-world domains.

\end{enumerate}

At the atomic level, the entities in the \eu plan representation are:

\begin{description}

\item[\textbf{Variables}] Values that need to be represented to
  describe the problem domain \kcomment{over which constraints are
    specified}.  In the Shopping Agent example, the times at which the
  agent needs to leave or be back, or executes a purchase, are all
  instances where \emph{variables} representing time is used.  As
  noted in Section \ref{sec:europa:cp}, a variable can take values
  from a domain. In \eu, domains are defined over a specific data
  type; the primitive data types supported are \textit{String},
  \textit{Boolean}, \textit{Numeric} and \textit{Object} (see
  below). Further, two types of variable Domains can be defined over a
  data type \textit{Enumerated} and \textit{Interval} (used only for
  numeric data types, they are specified by a
  $[LowerBound,UpperBound]$ pair, for instance: $[1,10]$,
  $[5.0,100.0]$). It is this latter ability to represent and reason
  over numeric intervals that forms the basis of representing flexible
  plans.

\item[\textbf{Constraints}] Valid plans in most problem domains have
  to satisfy a number of restrictions for instance, a Shopping Agent
  may only be able to perform one task at a time, may have limited
  time or an energy or fuel budget to perform a task. These
  restrictions are represented by \emph{constraints}, which can be
  defined over any combination of variables in a domain. \eu provides
  a \emph{Constraint Library} that implements a number of useful
  constraints (temporal, resource, relational) while allowing the
  possibility to \kcomment{augment} with new constraints if required.

\item[\textbf{Objects}] The items we wish to describe and refer to in
  a domain are considered objects.  In the Shopping Agent example, we
  might consider the Agent, Products and Locations all to be
  \emph{objects}. Objects have state and behavior.  For example, a
  Shopping Agent can have its location, a bag that contains the
  products it has already purchased (the bag can in turn be another
  object) and behavior, as in going to a location to look for a
  product, perform a purchase, return home, etc. all as objects.
  Those that have similar state and behavior can be described
  generically in terms of object types or classes. \eu allows the
  definition of classes similar to object-oriented programming
  languages, where a class can have attributes (represented by
  variables) and classes can be arranged in a single-parent hierarchy
  through inheritance.  \kcomment{What is different however, is that
    to describe state and behavior for planning one needs} to build on
  the formalism of first order logic.

\item[\textbf{Tokens}] In first order logic, a predicate defines a
  relation between objects and properties \cite{gen87}.  In \eu, we
  define such relations between variables whose domains are sets of
  objects and sets of properties to describe state and behavior. For
  example, one might use a predicate \texttt{At($a$,$l$)} to indicate
  that agent $a$ is at location $l$, or a predicate
  \texttt{Buy($a$,$p$}) to indicate that agent $a$ is acting to buy
  product $p$. Note that $a$ is a variable which may have a number of
  possible values in a problem with multiple agents. Similarly, $l$
  and $p$ are variables whose values are the set of possible positions
  and products respectively. \eu can be used to create partial plans,
  where the domains of variables $a$ and $l$ can have more than one
  possible value. Or they can be grounded \kcomment{representations},
  where single values will be specified for each variable as shown in
  the $N$-Queens problem.

  In general, for an executable plan it is insufficient to
  \kcomment{define} predicates that describe required state or
  behavior without also specifying some temporal extent over which
  each of those predicates hold. Predicates that are always true can
  be thought to hold from the beginning to the end of the planning
  horizon. However, in practice, the temporal extent of interest must
  be defined with timepoints to represent its start and end.  In such
  \kcomment{a} representation, one writes \texttt{At($a$,$s$,$e$,$l$)}
  to indicate that agent $a$ is at location $l$ from time $s$ to time
  $e$. In fact, this pattern of \kcomment{usage} to describe both
  state and behavior of objects is prevalent enough that \eu
  \kcomment{encapsulates into} the token
  representation. \kcomment{Tokens} have built in variables to
  indicate the object to which \kcomment{it primarily} applies as well
  as the timepoints over which it holds. A token, therefore, is an
  instance of a predicate that represents an object's state or
  behavior and is defined over a temporal extent. In \eu, all
  predicate instances are tokens. \kcomment{In} the same way that
  objects are described generically by classes, in \eu tokens can be
  described generically by token types. Every token has five built-in
  variables:

  \begin{enumerate}

  \item \textit {start}: The beginning of the temporal extent over
    which the token predicate is defined.

  \item \textit {end}: The end of the temporal extent over which the
    predicate is defined.

  \item \textit {duration}: The duration of the temporal extent. The
    constraint \textit{start} $+$ \textit{duration} = \textit{end} is
    enforced automatically.

  \item \textit{object}: The set of objects to which a token might
    apply. In a grounded plan representation each token applies to a
    specific object, reflecting the intuition that tokens are used to
    describe some aspect of an object (\ie state or behavior) in
    time. However, in a partial plan, it is possible that the
    commitment to a specific object may not yet have been made.

  \item \textit{state}: Tokens can be \texttt{ACTIVE, INACTIVE,
      MERGED}, or \texttt{REJECTED}. The state variable captures the
    token's current state and its reachable states through further
    restriction.  This is required to support \texttt{CAIP}'s approach
    to planning where goals and subgoals can be satisfied by either
    adding new intervals (tokens) to the plan, or by merging with
    existing intervals.
    
  \end{enumerate}

\item[\textbf{Built-in Object Types}] In a domain model, there may be
  some classes that don't have any time-dependent state or behavior,
  and therefore those classes will not have any token types associated
  with them. For instance in the Shopping Agent model, the set
  locations and products are static for the agent's purposes so one
  may only want to declare them but not associate any state or
  behavior with them.  However, a common case in a non-trivial domain
  model is that objects will have many associated tokens in order to
  describe their state and behavior throughout a plan. There are some
  object types that are used so often in domain descriptions that \eu
  provides a built-in implementation for them:

\begin{enumerate}

\item \textit{Timelines}: Often objects in a domain must be described
  by exactly one token for every given timepoint in the plan. Any
  instances of a class derived from \eus built-in timeline class will
  induce ordering requirements among its tokens to ensure no temporal
  overlap may occur among them \cite{mus94}. Fig. \ref{fig:europapr1}
  and others \kcomment{in this chapter,} illustrate how timelines are
  used to represent state in a plan.

\item \textit{Resources}: Metric resources, e.g. the energy of a
  battery or the capacity of a cargo hold are objects with an explicit
  quantitative state in time and with a circumscribed range of changes
  that can occur to impact that state i.e. produce, consume, use,
  change.  Resources are a common enough requirement for \eu users
  that built-in object types (with their corresponding token types to
  denote production, consumption, etc) are provided.  Instances of
  classes derived from a resource will induce ordering requirements on
  their tokens in order to ensure that the level of the resource
  remains within specified limits.

\end{enumerate}

\item[\textbf{Token State Model}] In \eu's representation, goals are
  temporally scoped states that need to be achieved, or actions that
  must be performed. As a result, \eu internally represents goals as
  tokens.  Goals can be posted explicitly, for example, when a user
  asks the Shopping Agent to own a specific product by a specific
  time, or implicitly when subgoals are created as a consequence of
  domain rules. For instance, the Shopping Agent must get to a
  location where a product is available before it is able to purchase
  it. In this case the original goal of purchasing the product spawns
  a \emph{subgoal} of being at a specific location without which the
  purchase will not occur as defined in the domain model.  

  Looking at how a simple Shopping Agent goal would be addresed by \eu
  helps illustrate how Token States support the planning process.
  Let's assume the planner starts with a partial plan that places the
  Shopping agent at home at time $0$ with no products in
  possession. Let's also assume that goals are posted for the Agent to
  own a drill by time $10$ and to be home by time $20$ as shown in
  Fig. \ref{fig:europapr1}
  
  \begin{figure}[t]
    \centering 
    \subfloat[\small Timelines representing initial state
    and goals for a Shopping Agent.]{\label{fig:europapr1}\includegraphics[scale=0.4]{figs/europa-pr-1.pdf}}\qquad
    \subfloat[\small State of timelines after satisfying a goal to be
    home by time $20$ on merging with initial state.]{\label{fig:europapr:merge}\includegraphics[scale=0.4]{figs/europa-pr-merge.pdf}}\qquad
    \subfloat[\small State of timelines after satisfying a goal to own
    a drill by time $10$ by activating corresponding
    token.]{\label{fig:europapr:activate}\includegraphics[scale=0.4]{figs/europa-pr-activate.pdf}} 
    \caption{\small Evolution of a timeline from the Shopping Agent example.}
  \end{figure}


  Note that the tokens \texttt{Own(Drill)} and \texttt{At(Home)} that
  represent the goals are not immediately placed on the
  \texttt{AgentBag} or \texttt{AgentLocation} timelines, since \eu
  must first determine whether it can achieve those \kcomment{goals}
  and how. When the goals are posted and before any decisions are
  made, the \texttt{Own(Drill)} and \texttt{At(Home)} tokens are
  created in an \texttt{INACTIVE} state. \eu then considers the
  partial plan and evaluates two possible alternatives:

\begin{enumerate}

\item If there are any tokens in the plan that are compatible with the
  new goal, \eu may attempt to satisfy the goal by \emph{merging} it
  with any of them (if there is more than one it can try them one at a
  time or in some heuristic order), in this case, the state of the
  token would change to \texttt{MERGED}.

\item Instead of merging with existing tokens in the plan, \eu may
  decide to \emph{insert} a new token into the plan, in which case the
  state would change to \texttt{ACTIVE}.

\end{enumerate}

When a token is merged, all restrictions imposed on the merged token
are transferred to the active token upon which it is merged. Merging a
token requires finding a target \texttt{ACTIVE} token that is
compatible. Two tokens are merge-compatible if they are instances of
the same predicate and there are no empty intersections between
corresponding token variables.  The effect of merging is illustrated
in Fig. \ref{fig:europapr:merge}. The goal to be home by time $20$ is
satisfied by merging with the initial state that \kcomment{indicated}
that the Agent is home at time $0$.  The merge is possible because the
tokens are of the same \texttt{At} type and intersecting the domains
for all of their variables does not yield an empty domain (representing
an inconsistency). In this case the token variables are the location,
which is \texttt{Home} in both cases and the temporal variables start,
end and duration, which can be intersected without causing an
inconsistency. After the token representing the goal is merged onto
the token representing the initial state, the resulting token
represents not only that the Agent must be \texttt{At(Home)} by time
$0$ but also \texttt{At(Home)} at time $20$, This is reflected in the
new constraints on the token's start and end variables as depicted in
Fig. \ref{fig:europapr:merge}.

When a token is activated, all domain rules associated with its type
are evaluated, which may lead to new subgoals being generated. In our
example, there could be a domain rule stating that if a Shopping Agent
doesn't already own a product, then it must \texttt{Buy} it before it
can \texttt{Own} it. In such a case, activating the
\texttt{Own(Drill)} token will cause a new \texttt{Buy(Drill)} to be
created in an \texttt{INACTIVE} state, which in turn could spawn its
own subgoals, like being at the location where the product can be
purchased before being able to buy it, and so on, until all domain
rules are satisfied. As Fig. \ref{fig:europapr:activate} shows the
decision to merge the goal to be \texttt{At(Home)} by time $20$ with
the initial state was not appropriate since the Agent will have to get
away from \texttt{Home} to be able to buy the drill which is
inconsistent with the constraints that resulted after the merge. As a
result that decision will have to be retracted and the alternative to
activate or merge with a different token will need to be considered.

If after looking at all possibilities, neither activating nor merging
work, a token may be \texttt{REJECTED}. This is possible only for
goals posted by the end user and specified as optional. Subgoals
generated as a result of token activation cannot be rejected. If a
subgoal cannot be satisfied through activation or merging then it will
cause the original activation decision that spawned it to be
retracted.

As can be seen from the above example, during the search for a plan, a
token's state can go from \texttt{INACTIVE} to \texttt{ACTIVE, MERGED}
or \texttt{REJECTED} as shown in Fig. \ref{fig:europapr:states}.

\begin{figure} \centering
  \includegraphics[scale=0.45]{figs/europa-pr-states.jpg}
  \caption{\small Possible transitions for a token state.}
\label{fig:europapr:states}
\end{figure}


\item[\textbf{Rules}] In order for a plan to be valid, it must comply
  with all rules pertinent to the relevant application domain. Rules
  govern the internal and external relationships of a token. For
  example, consider a parameterized action
  \texttt{Go(origin,destination)} which describes a Shopping Agent
  traveling from one location (\textit{origin}) to another
  (\textit{destination}). The parameters are instantiated on a token
  as variables whose domain of values is the set of all possible
  locations in a given problem. A rule governing an internal
  relationship among token variables might stipulate that a transition
  must involve a change in location. This can be expressed as a
  constraint on the definition of a predicate of the form:
  \textit{origin != destination}. A further stipulation could be that
  the agent must be located at \textit{origin} before travel can occur
  and the agent must be located at \textit{destination} when travel is
  completed. This is an example of a rule governing an external
  relationship among tokens. It specifies a requirement that tokens of
  the predicate \texttt{At} that represent the location state for the
  Shopping Agent precede and succeed tokens of the predicate
  \texttt{Go} that represents traveling. Fig. \ref{fig:europapr:rules}
  illustrates the entities and relations involved in specifying such a
  rule on a \texttt{Go} action. The token on which the rule applies
  (\texttt{Go}) is referred to as the \emph{master}. Each \texttt{At}
  and \texttt{Going} token required by the master is referred to as a
  \emph{slave}.  Application of a rule on a token \kcomment{(\ie a
    master)} can thus cause slave tokens, variables and constraints to
  be introduced.  The capability of a domain rule to cause a slave
  token to be created is a key vehicle through which planning
  occurs. Semantically, a rule imposes a requirement for supporting
  slave tokens to be in the plan in order for the master token to be
  valid.

  \begin{figure}[t]
    \centering
    \includegraphics[scale=0.45]{figs/europa-pr-rules.pdf}
    \caption{\small Master and Slave tokens generated as a result of a domain rule.}
    \label{fig:europapr:rules}
  \end{figure}

\end{description}

%%% Local Variables: 
%%% mode: latex
%%% TeX-master: "ieee-ram09"
%%% End: 
