\section{Conclusion}
\label{sec:conclusion}

% Final parting words summarizing the contribution of deliberation towards
% vehicle autonomy.

The Ocean Sciences are at a cusp, straddling between traditional
ship-based exploration with newer observatory-based methods. In the
United States the NSF funded Ocean Observatories Initiative \cite{ooi}
and regional bodies such as \texttt{CeNCOOS} \cite{cencoos} are
investing in new observational methods and technologies which propose
to aid oceanographers by producing substantially higher-resolution
data. These will use moorings, cabled observatories and glider fleets
generating near real-time data from one or more locations as a
continuous stream. The tack chosen by \can has been to complement such
methods with the aim of developing methods for a \emph{portable
  mobile} observatory which is dependent on autonomous platforms. In
this context what we advocate in this chapter is an initial step in
making robotic platforms more adaptive, scaling them towards fleets of
vehicles (underwater or on the surface) and most importantly using
them to solve the larger challenge of Sampling spatio-temporal fields.

While our trajectory of research in the oceanographic domain follows
that in spacecraft autonomy, from onboard \cite{mus98} to
mixed-initiative methods \cite{bresina05}, the challenges are
significantly different with the environment playing a far larger
role. Moreover, prediction and projection of a future course of action
for a robot given the level of uncertainty calls for representational
and architectural methods which would enable adaptation at different
levels of abstraction, from mission-planning to low level actuation. 

Our efforts were initially targeted to full robotic autonomy; while
this is an important goal that we continue to strive towards, it is
increasingly clear that human-in-the-loop methods also have a sizable
role to play in ocean exploration and by extension to Marine
Robotics. Not only is an embedded \rx an important element of the
effort towards smarter vehicles, but we are working towards an
implementation of \rx which will perform multi-vehicle plan synthesis
for the \od.

A key role that \rx has played and will continue to play is towards
\emph{event response} scenarios. In Section \ref{sec:results} we show
the role Machine Learning (ML) techniques have played thus far in
in-situ plan adaptation. This interaction between Planning and ML we
believe will play a significant role going forward in the general
theme of exploration and discovery. Interpretation of incoming sensor
data to deal with scientific surprise and readapt for opportunistic
science will be critical to dealing with persistent problem of
undersampling the ocean\footnote{Walter Munk of the Scripps Institute
  of Oceanography has famously stated \emph{``Most of the previous
    century could be called a “century of undersampling”''} --
  Testimony to the U.S. Commission On Ocean Policy, 18 April
  2002}. Much of what we know of the ocean was derived by exploring in
the blind. We anticipate longer duration robotic vehicles with enough
onboard computational capacity to run deliberative agents such as \rx
make water-column measurements and reliably track and characterize
dynamic features. While our experiments have already shown such
capabilities the exploration Vs. exploitation tradeoff is better
determined when vehicles can make choices over longer time periods
(weeks and months) to study the impacts of bio-geochemical interaction
driven by longer duration physical forcing.

Given our poor understanding of the coastal ocean in particular,
mixed-initiative methods of command and control for Sampling with
heterogenous assets is an important goal that is clearly at our
doorstep. We believe deliberation is but a small yet critical part of
the solution to unfolding open research problems for AI and Marine
Robotics.
