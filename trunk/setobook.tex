\documentclass[12pt]{svmono}

% \usepackage{fullpage}
\usepackage[pdftex]{graphicx}
\usepackage[lofdepth,lotdepth]{subfig}
\usepackage{floatflt}
\usepackage{float}
% \usepackage{epsf}
% \usepackage{epsfig}
% \usepackage{subfigure}
\usepackage{subfig}
\usepackage{latexsym}
%\usepackage{algorithm}
%\usepackage[noend]{algorithmic}
\usepackage{color}
\usepackage{wrapfig}
\usepackage{topcapt}
\usepackage{multirow}
\usepackage{tabularx}
\usepackage{amsmath, amsfonts, amssymb}
% \usepackage{algorithmic}
\usepackage[hmargin=1in,vmargin=2.5cm]{geometry}
%\usepackage{url}
% \usepackage{multirow}
\usepackage[ruled,noline,linesnumbered]{algorithm2e}
% \usepackage{soul}
% \usepackage{alltt}
%\usepackage[caption=false]{caption}

\widowpenalty=1000
\clubpenalty=1000

\def\rx{{\texttt{T-REX\ }}}
\def\eut{{\texttt{EUROPA}$_2$\ }}
\def\eu{{\texttt{EUROPA}\ }}
\def\eus{{\texttt{EUROPA}'s\ }}
\def\etal{{et al.\/}}
\def\eg{e.g., }
\def\ie{{i.e.,\ }}
\def\etc{{etc.\ }}
\def\situ{{in situ \/}}
\def\PN{{\emph{PN} }}
\def\can{{\texttt{CANON\ }}}
\def\od{{\texttt{ODSS\ }}}
\input{epsf}

\usepackage[normalem]{ulem}

%\usepackage{mathptmx}
%\usepackage{multirow}

\newcommand{\rtime}[1]{\par\noindent\rlap{#1} \hspace*{2.15cm}}
\newcommand{\iblank}{\par \noindent \hspace*{2.4cm} \hangindent 2.6cm}
\newcommand{\m}[1]{\ensuremath{\mathbf{#1}}}
\newcommand{\mc}[1]{\ensuremath{\mathcal{#1}}}
\newcommand{\mb}[1]{\mbox{\boldmath$#1$\unboldmath}}
%\newcommand{\norm}[1]{\left| \left| #1 \right| \right| ^2}
\newcommand{\snr}{\hbox{SNR}}
\newcommand{\mse}{\hbox{MSE}}
% \newcommand{\E}{{\mathbb E}}
\newcommand{\cn}{{\mathcal{CN}}}
\newcommand{\ba}{\begin{align*}}
\newcommand{\ea}{\end{align*}}

\newtheorem{Prop}{Proposition}
\newtheorem{Theorem}{Theorem}
\newtheorem{Lemma}{Lemma}
\newtheorem{Corrolary}{Corollary}

\def\be{\begin{equation}}
\def\ee{\end{equation}}

% \newlength{\doublespacelength}
% \setlength{\doublespacelength}{\baselineskip}
% \addtolength{\doublespacelength}{0.5\baselineskip}
% \newcommand{\doublespace}{\setlength{\baselineskip}{\doublespacelength}}

% \newlength{\singlespacelength}
% \setlength{\singlespacelength}{\baselineskip}
% \newcommand{\singlespace}{\setlength{\baselineskip}{\singlespacelength}}


% \newlength{\savedspacing}
% \newcommand{\savespacing}{\setlength{\savedspacing}{\baselineskip}}
% \newcommand{\restorespacing}{\setlength{\baselineskip}{\savedspacing}}

% \setlength{\parskip}{0pt}
% \setlength{\parsep}{0pt}
% \setlength{\headsep}{0pt}
% \setlength{\topskip}{0pt}
% \setlength{\topmargin}{0pt}
% \setlength{\topsep}{0pt}
% \setlength{\partopsep}{0pt}

\newcommand{\kcomment}[1]{{\color{red}{KR: #1}}}
\newcommand{\acomment}[1]{{\color{green}{AO: #1}}}
\newcommand{\jcomment}[1]{{\color{cyan}{JD: #1}}}
\newcommand{\fcomment}[1]{{\color{red}{Fp: #1}}}
\newcommand{\comment}[1]{{\color{blue}{#1}}}

\setcounter{secnumdepth}{3} 
%\linespread{0.95}
 % \linespread{2.00}


\usepackage{graphicx}

\usepackage[agsm]{harvard}

\title{\textbf{\sc Deliberative Autonomy for Marine Robotics}}
\author{}
\date{\small Monterey Bay Aquarium Research Institute \\
7700 Sandholdt Rd.\\
Moss Landing,  CA, 95039\\
{\tt }}

\begin{document}

\maketitle

\begin{abstract}

  We describe a general purpose Artificial Intelligence based control
  architecture that incorporates in-situ decision making for
  adaptation and control for autonomous underwater vehicles
  (AUVs). The Teleo-Reactive EXecutive (\texttt{T-REX}) framework
  deliberates about future states, plans for actions and executes
  generated activities while monitoring plans for anomalous
  conditions. Plans are no longer scripted a priori but synthesized
  onboard with high-level directives instead of low level
  commands. Further, the architecture uses multiple control loops for
  a 'divide and conquer' problem solving strategy allowing for
  incremental computational model building, robust and focused failure
  recovery, ease of software development and ability to use legacy or
  non-native computational paradigms. Vehicle adaptation and sampling
  occurs in situ with additional modules which can be selectively used
  depending on the application in focus. Abstraction in problem
  solving allows different applications to be programmed relatively
  easily, with little to no changes to the core search engine thereby
  making software engineering sustainable. The representational
  ability to deal with time and resources coupled with Machine
  Learning techniques for event detection allows balancing shorter
  term benefits with longer term needs, an important need as AUV
  hardware becomes more robust allowing persistent ocean sampling and
  observation. \rx is in operational and routine use at MBARI,
  providing scientists a new tool to sample and observe the dynamic
  coastal ocean.
  
%  % Traditional approaches to commanding marine robots have relied on
%  %  pre-scripted plans with a priori defined waypoints. These have
%  %  served the oceanographic community well in enabling robots to
%  %  cost-effectively perform surveys traditionally undertaken by
%  %  ship-board observations. However, in dynamic coastal environments
%  %  where is substantial interaction between atmospheric, oceanographic,
%  %  estuarine/riverine and land-sea interaction processes requires a
%  %  robot to be adaptive and receptive to opportunistic
%  %  science. Sustained use of such autonomous platforms in military
%  %  applications have also pushed.

\end{abstract}

\section{Introduction}
\label{sec:intro}

% Talks about how AUV missions are traditionally planned. Why they might have
% been adequate in the past and are no longer so. How can deliberation
% (i.e. planning help) and how can it contribute to new ways of observing the
% ocean.

Autonomous platforms in the marine environment have had a substantial
and immediate impact for both civil and military
applications. Autonomous Underwater Vehicles (AUVs) have not only
altered the ability of oceanographers to reach beyond the surface and
make high resolution observations, but have impacted the engineering
methodologies behind sampling, control and robotics itself
\cite{Brierley08032002,ryan05,Thomas06,Yoerger01012007,Incze2009,Rigby10}. Yet
challenges remain, principally in making such robotic devices more
attune to the environment, to sample large swaths at the Meso-scale
($> 50$ km\textsuperscript{2}) and doing so systematically and
adaptively. Further, over the last two decades substantial
improvements in hardware including propulsion and battery technologies
have outpaced software especially in algorithmic methods in sampling
and control and in systematic software engineering development. Large
scale high resolution observations have become increasingly tractable
using autonomous platforms both on and below the surface.  This, in
turn, has lead to periodic calls for sustained exploration and
sampling \cite{rudnick03}, illustrating the high scientific value
placed on such data.  The complexity of oceanic processes and the
chronic under-sampling of the coastal oceans, especially of dynamic
features studied over a variety of spatial and temporal scales, has
made it challenging for robotic platforms to autonomously track and
sample. A prominent example of such a process is coastal algal blooms
which are patchy and could cover large coastal zones. Persistent
observation of such dynamic events dictates that our robotic assets
track and sample such patches which can evolve rapidly due to inherent
bio-geochemical activity, as well as advection and diffusion of the
water mass it resides in.

At the Monterey Bay Aquarium Research Institute (MBARI), a unique
inter-disciplinary collaboration between biologists, ecologists,
geneticists and roboticists on the \can (Controlled Agile Novel
Observation Network) program \cite{canon} has driven autonomous system
exploration and control. Fundamental questions related to sampling
strategies have driven the use cases for advanced concepts in
autonomy, specifically to Artificial Intelligence (AI) Planning and
Execution. Current work is also looking at how other fields in Machine
Intelligence especially Machine Learning can be leveraged to impact
the science of engineering systems in marine robotics and as a
consequence oceanography itself.

In this chapter, we propose an alternative view to decision-making
using in-situ automated Planning \cite{ghallab04}. More importantly,
we demonstrate that earlier control paradigms (primarily driven by
\emph{reactive} approaches to control) can be augmented to provide
targeted observation and sampling as well as a more nuanced and
balanced consideration of mission objectives, environmental conditions
and available resources. Planning (or deliberation or projection in
action space\footnote{\kcomment{We use the terms 'planning' and
    'deliberation' interchangeably in this chapter.}}) is important to
balance current needs of a robot with future desires or goals taking
into account time and available resources in-situ on the
robot. Planning without time or action planning in and of itself is
not adequate; dealing with time ensures that the balance between now
and the future is handled systematically. Yet, plans have to be tied
to robotic action and in dynamic real-world
\kcomment{environments}. Further, plans \kcomment{not only} project
future state which are likely to change, but \kcomment{need to factor
  in} substantial environmental uncertainty\footnote{It is important
  to remember that these principles are broad much as Dwight
  Eisenhower is reputed to have said ``\emph{In preparing for battle I
    have always found that plans are useless, but planning is
    indispensable}'' and ``\emph{Failing to plan is planning to
    fail}''.}. \kcomment{In the oceanographic context,} use of
environmental cues is often hampered by poor \kcomment{predictive}
skill \cite{anderson2009} of ocean models and little to no
availability of a priori synoptic views of the survey
area. \kcomment{Planning therefore provides a ``buffer'' against the
  stochastic environment.}


\kcomment{To date} AUV control architectures have primarily been
driven by reactive Subsumption-based approaches \cite{brooks86}. They
have been adequate for routine straight-line survey trajectories which
have played a valuable role in the acceptance of AUVs as a new tool
for ocean science and \kcomment{military} applications. However,
scaling of problem domains in addition to dealing with event-response
and discovery in the water-column, challenges \emph{how} controllers
are written, maintained and upgraded over the life-cycle of the
platform. Dealing with off-nominal endogenous or exogenous conditions
including sensor failures stretches how AUVs are currently being
utilized; one consequence for instance, is any off-nominal condition
results in a \emph{fail-safe} \kcomment{surfacing} mode to radio for
help. Software design using these reactive approaches, not only makes
it hard for sustaining engineering development, but these approaches
are not adequately adaptive for emerging ocean science problems.
Despite vigorous defense by Brooks early on
\cite{Brooks91intelligencewithoutrea,Brooks91intelligencewithoutrep},
autonomous systems do require a model if not of the world then of the
\emph{physics} of the vehicle when it interacts with the world.


Criticism of AI planning, on the other hand, also well articulated by
Brooks, was targeting the performance of in-situ
deliberation. Terrestrial robotic platforms were well-known to
frequently stop and ``plan'', assuming in the meantime that the world
would remain static \cite{shakey84}. In the marine robotics world,
such a situation is not permissible given the need to make continuous
observations without stopping a \kcomment{non-holonomic} robotic
platform such as an AUV. However, algorithmic and implementational
advances by the AI community over the years have produced a range of
planners with demonstrated embedded capabilities
\cite{simmons94,Haigh98,alami:1998p820,chien00,mus98,teichteil07}. Our
work has been influenced by and in turn influenced a wide range of
activities in the field of AI Planning and Plan Execution with
demonstrated real-world capabilities in the Space domain
\cite{mus98,rajan00,aichang04,bresina05} for NASA. Lessons learned
have been applied to the world of marine robotics which we bring
together in this chapter. Important lessons that were derived from
this rich legacy included the following:

\begin{itemize}

\item Projecting plans (or using \emph{deliberation}) can and should
  balance near-term objectives with end-term or even evolving goals.

\item Dealing with time and resources is essential in real-world
  planning.

\item A rich representation that can deal with co-temporal events for
  modeling complex systems is \kcomment{important}.

\item Fast solvers for incremental causal reasoning can be built as a
  basis for dealing with dynamic controllable events where replanning
  is tightly integrated.

\item Planning should be tightly inter-leaved with execution to enable
  responsiveness to events which impact plan failure. This is also
  important \kcomment{for} platform adaptivity.

\item A model-based approach should separate control formulation of
  the platform which describes operational characteristics, from the
  domain. The controller can then be rigorously tested and certified
  even as the domain model evolves to fit \kcomment{evolving
    requirements and/or} diverse applications.

\end{itemize}

We have developed, tested and deployed the Teleo-Reactive EXecutive
(\rxe) an on-board adaptive control system that integrates AI based
planning and state estimation in a hybrid executive
\cite{mcgann08a,mcgann08b,py10}.  State estimation allows the system
to keep track of world evolution as time advances based on a formal
model distributed through the architecture.  Onboard planning and
execution enables adaptation of navigation and instrument control
based on \kcomment{estimated state}.  It further enables goal-directed
commanding within the context of projected mission state and allows
for replanning for off-nominal situations and opportunistic science
events. The framework in addition to being used on an AUV, is general
enough \kcomment{to} be used for controlling a personal robot
\cite{pr2,Meeussen:2010dn,mcgann2009} and is deployed on a European
planetary rover testbed \cite{goac11}. While it is possible to
integrate probabilistic estimation on top of this framework
\cite{mcgann08d}, such representation is not explicitly handled by the
framework and is out of scope of this chapter.  Deliberation and
reaction are integrated systematically over different temporal and
functional scopes within a single agent and a single model that covers
the needs of high-level mission management, low-level navigation,
instrument control, and detection of unstructured and poorly
understood phenomena. \rx is deployed on MBARI's \emph{Dorado} AUV
shown in Fig. \ref{fig:auv-fig}, which to the best of our knowledge is
the only operational marine platform anywhere being used for routine
scientific surveys with onboard plan synthesis.

\begin{figure}[t]
  \centering \vskip-5pt
  \includegraphics[scale=0.1]{figs/MBARI-AUV.jpg}
  \caption{\small The MBARI \emph{Dorado} AUV on its support vessel
    the R/V \emph{Zephyr}.}
  \label{fig:auv-fig}
  \vskip-0.3cm
\end{figure}

This chapter is organized as follows. We start \kcomment{by placing
  this effort in the context of the overal research in AI Planning in
  Section \ref{sec:related}} followed by some foundational concepts in
our Planning framework in Section~\ref{sec:concepts} which includes a
brief overview of AI Planning in Section~\ref{sec:planningfound}. In
Section~\ref{sec:basics} we go deeper into the \eu framework itself,
the basis of all deliberation within \rxe. A key portion is
Section~\ref{sec:europa:pr} which details the internal plan
representation within a situated agent like \rxe. Modeling
(Section~\ref{sec:europa:modeling}), Inference
(Section~\ref{sec:europa:inference}) and Search
(Section~\ref{sec:europa:search}) articulate important details
associated with \eu critical to \rxe.  With these foundational
elements of deliberation out of the way, we transition to \rx itself
in Section~\ref{sec:arch}, bring out architectural details in
Section~\ref{sec:arch:trex}, deal with the execution cycle in
Section~\ref{sec:arch:exec}, highlight how \rx deliberates in
Section~\ref{sec:arch:europa}, finally delving deeper into the key
concept of synchronization in Section~\ref{sec:arch:synch}. Further
details on deliberation within \rx (Section~\ref{sec:arch:plan}) and
how planning and execution are interleaved are detailed in
Section~\ref{sec:arch:intertwine}. We offer results from our
experiments at sea in Section~\ref{sec:results}, offer some insights
into the near-term future goals of our research in
Section~\ref{sec:future} and conclude with
Section~\ref{sec:conclusion}.

%%% Local Variables: 
%%% mode: latex
%%% TeX-master: "setobook"
%%% End: 


\section{Background and Related Work}
\label{sec:related}

% Refer to past work in AUV autonomy as well as robotic autonomy in
% general. This is general purpose and should be a good literature survey
% for robotics and AI.

A distinct problem in Marine Robotics has been the use of the
overloaded term 'autonomy'. Not only does the notion transcend
different disciplines within engineering in this domain (e.g. the
Control Engineering community sees it distinctively from those in AI)
but users of marine platforms in oceanography as well as in maritime
defense conflate the methodology of use (tethered vs. untethered) with
their control. In this chapter, our notion of 'autonomy' deals
explicitly not only with the notion of dealing with the \emph{sense,
  plan and act} paradigm, but also the need to perform
\emph{computational search} between different action outcomes, an idea
central to the field of AI. Consequently, our literature survey here
is targeted towards a more focused and (to this chapter) relevant part
of the field of computation.

With few exceptions, most AUV control systems have been variations of
reactive approaches.  Typically, waypoint-based pre-defined mission
designs are uploaded to the AUV; specialized code fragments called
\emph{behaviors} are designed for the specific mission and a choice of
behaviors for the mission are used on the computational stack
\cite{bellingham94}. Swapping in and out of these behaviors using
conditionals or cleverer and quantitative approaches such as
\cite{Benjamin:2004} forms the basis for adaptation and safety in the
vehicle. Where such swapping does not aid adaptation, individual
behaviors are tweaked to generate some measure of responsiveness to
the surrounding environment \cite{yanwu08} to generate interesting
observations for oceanographers, yet technically naive in terms of
scalability and systematicity. \cite{Benjamin2006} is clear about the
\kcomment{extent} of adaptation in stating:

{\footnotesize
  \begin{quote}
\small \emph{The primary difficulty often associated with behavior-based control
concerns action selection - namely how to ensure the chosen action
really is in the best overall interest of the robot or vehicle.}
\end{quote}
} 

Such techniques have proved their mettle in the first round of use of
AUVs; they've provided operators a simple way to handle control
complexity of the vehicle while returned data at far higher resolution
than would have been possible using traditional ship-based methods and
at substantially reduced amortized costs of platform operation and
ownership over multiple years. AUV operators and clients for their
data products have reasons to be well satisfied with results thus
far. However, as hardware and more cost-effective sensors become more
robust to the environment and as mission durations increase (as
demonstrated by glider experiments such as \cite{rucool11}), sustained
presence in the ocean requires the ability to deal with off-nominal
conditions, that balances the needs of high-level mission management,
low-level navigation, instrument control and detection of unstructured
and poorly understood phenomena. Earlier reactive methods are unable
to perform such a balancing act without skewering the foundations of
the software development methodology leading to code bloat or worse.
\cite{carreras06} provides a reasonably good overview of AUV control
architectures that essentially is based on reactive control methods.

\begin{figure}[htpb]
 \centering
 \includegraphics[scale=0.3]{figs/planner-arch.pdf}
 \caption{\small A general architectural block diagram for an AI based
   Planner.}
 \label{fig:planner}
\end{figure}

A word about generalized planning. Traditionally planning has been
considered computationally intensive (the general planning problem can
be worse than NP-complete \cite{ghallab04}). A major reason for this
belief had to do with the role of the planner (which was assumed to be
generative) and its place in the architecture (infrequently called to
re-plan the \emph{entire} mission in off-nominal conditions) within a
\emph{sense-plan-act} paradigm. This can limit the reactivity of a
robot especially when the environment could change at a rate faster
than the planner can plan.  

The architecture of a typical AI planner is along the lines of
Fig. \ref{fig:planner}. Given a \emph{domain model} which encodes the
platform constraints in a higher-level language, an \emph{initial
  state}, a high-level objective or end \emph{goal}, the \emph{plan
  database} (or \texttt{plandb}) is where all assertions are tracked
and maintained. Theorem proving in the form of satisfaction of axioms
embodied in the domain model occurs within the \texttt{plandb}. It is
the job of the \emph{search engine} to provide the inferential
mechanism for state space exploration potentially aided by heuristics
encoded for \emph{search control}. In some complex domains
\cite{mus98} \emph{external modules} can provide additional domain
expertise which can augment the domain model. The end result is plan
formulation using a multitude of approaches either using forward or
backward search or using more generalized \emph{partial order} methods
\cite{ghallab04}.

Deliberative techniques for robotic control on the other hand
contrast in providing a distinct set of advantages:

\begin{enumerate}

\item action selection is driven axiomatically based on a systematic
  assessment of a range of causal conditions. An action is selected
  for instantiation only when there is causal support in the form of a
  constraint in a deterministic model of the vehicle's operation.

\item the system is goal directed which forms an objective towards
  which the system is expecting to converge providing a foci in action
  selection.

\item scaling to different (computational) problem size is dependent
  on incremental model building rather than being concerned about
  nuanced interactions between behaviors.

\item systematic software engineering methodologies such as spiral
  development \cite{boehm86} can further be used for scaling up the
  task as demonstrated by \cite{DS1report}.

\item when interactions between actions and/or behavior fragments do
  (and must) occur there are explicit constraints encoded in the model
  that must be computationally satisfied contrasting with carefully
  crafted strategies to instantiate behaviors on a stack. Such
  explicit recording of constraints in the model in turn aids
  long-term software maintenance and sustaining engineering.

\end{enumerate}

There are costs associated with such inference based control which are
not inconsequential. Foremost among them is the steep learning curve
that comes with the design of the domain model. With limited support
tools necessary for knowledge capture and design of the constraints
often in stylized higher-level languages (see \cite{NDDL} for an
example), it requires the modeler to think in alternative ways to do
problem solving. More importantly it is often necessary for the
modeler to be exposed to the internal representation of the planner
and how it performs search. The level of expertise is often well above
what would be expected of a typical well-rounded Computer Science
graduate student. Yet in our experience, the cost of model design in
such systems, is well balanced against the cost of re-engineering new
mission scripts for new deployment in more ``classical'' script-based
controllers.

However this is mild criticism given the end benefits and the
scientific and engineering goals associated with adaptive and
persistent observation for marine robots whether for civil or military
applications. Given the general problem-solving nature of such
deliberative systems, there is also a substantial overall benefit for
the field of robotics. 

\subsection{Evolution of AI-based Robotic Planning: A Necessary Digression}
\label{sec:related:robotplans}

Planning and plan execution are not new to the overall field of
robotics. Early motivation of building computational mechanisms for
decision making were intended to be deployed on robotic devices by
very definition. The seminal volume \cite{computersthought} very
clearly articulates how physical manifestation of robots could be
controlled by task planning and execution. The very first planner
\cite{green69} was an exercise in computational theorem proving with
early state space planners popularized by \texttt{STRIPS}
\cite{strips71} dominating the academic landscape. An early
demonstration of planning as a situated agent was the highly
influential \texttt{Shakey} the robot experiment
\cite{shakey84}. Subsequent work driven largely by defense funding in
the United States sequestered the discipline in academic labs
principally applied for algorithmic development and if embedded then
in testing within the confines of modest indoor settings. The leap to
applying them for real world problem solving is relatively new and
driven largely by NASA applications pioneered by \cite{mus94,mus98,
  jonsson00, rajan00, chien05, bresina05}.

The dominant approach for building agent control systems utilize a
$3$-layered architecture (see \cite{gat98} for a well thought out
rationale), notable examples of which include \texttt{IPEM}
\cite{AmbrosIngerson88}, \texttt{ROGUE} \cite{Haigh98}, the LAAS
Architecture \cite{alami:1998p820}, the Remote Agent Experiment
\texttt{RAX} \cite{mus98} and the Autonomous Spacecraft Experiment
\texttt{ASE} \cite{chien99} (see \cite{Knight01} for a survey). % Early
% efforts were cognizant of if absent with, dealing with time and
% resources for such experiments.

Representationally, the first paper on managing time systematically
within AI Planning frameworks was \cite{dean87}; subsequent efforts by
Boddy and others \cite{Dean88,Boddy93} refined the notion of using a
temporal database and the conceptualization of temporal
intervals. Using these base-level concepts Muscettola and others
\cite{mus92, ghallab94, laborie95, cesta96} developed the notion of
plan/schedule envelopes using the notion of state variable
instantiated timelines. Simultaneously work by Dechter et.al
\cite{dechter91} resulted in efficient temporal constraint propagation
which systematically defined the notion of Simple Temporal Networks
(STNs) and path and arc consistency algorithms. This work neatly put
together earlier efforts by Mackworth and Freuder \cite{mackworth77,
  mack85} and the Constraints community on propagation algorithms. In
parallel work in the UK as a result of US DARPA funding, resulted in
the design of \texttt{O-Plan} \cite{currie91}, which leveraged the
notion of constraints and metric time, but within a Hierarchical Task
Network (HTN) representation. All these efforts were soon after Vere’s
paper on planning and time \cite{vere83} which had a profound effect
on the AI Planning community. In particular the work of Muscettola
\cite{mus94} was embraced by NASA for telescope scheduling. Around the
same time period, researchers at LAAS came up with an architecture
\cite{alami:1998p820} which encapsulated the \texttt{IxTeT} planner
\cite{ghallab94}. It featured advanced concepts for a temporal planner
with the notion of chronicles, plan recognition for partial plan
fragments and early use of systematic resource
constraints. Development on \texttt{IxTeT} has continued albeit at a
slower pace with the contribution of various PhD students’ thesis.

\subsection{The Remote Agent Experiment and beyond}
\label{sec:rabeyond}

While \texttt{HSTS} \cite{mus94} was being implemented as a
ground-based planning tool for decision support for US based space
observations (EUVE and Cassini) \cite{mus95}, the opportunity to fly
onboard NASA’s New Millennium Deep Space $1$ came about. The design of
the Remote Agent Experiment \texttt{RAX} \cite{pell97, bernard98,
  pell98, mus98, DS1report, rajan00, jonsson00} was a direct off-shoot
of this effort where \texttt{HSTS} (written in LISP) was flown on
board and successfully commanded the spacecraft for a week in May
1999. There were a number of significant software engineering lessons
learned with the HSTS technology as integrated for the RAX:

\begin{enumerate}

\item Constraint-based representations were not only sufficient for
  plan synthesis, but valuable during debugging and development as a
  means of building a viable domain model.

\item Domain models when separated from the search engine as
  articulated by the model-based approach \cite{williams96a} ensured
  that sufficient effort would be targeted on human validation of the
  model building phase while ensuring that search engine stability
  across applications and domains resulted in lower cost to deploy.

\item Flexible temporal representation generating a range of plans as
  against a single plan, were robust for plan execution.

\item If planning and execution were disconnected (as in
  \texttt{RAX}), dispatchabilty \cite{mus98a} and controllability
  \cite{morris00} issues within temporal plans need be addressed. In
  \texttt{RAX}, a post-processing step was added to counter these
  effects. It was clear that execution needed access to the planners
  database and be able to propagate execution time constraints prior
  to command dispatching.

\end{enumerate}

It was with the demonstration of \texttt{RAX} $65$ Million miles in
outer space, that temporal reasoning using methods came to the
forefront of AI planning and for situated agents. Lessons learned from
\texttt{RAX} led to the development of \texttt{IDEA}
\cite{mus02,mus04,Dias:2003ua,mus06} with the central theme of using
planners collectively for problem solving arose. Another critical
theme was iteratively and incremental plan repair of an anytime plan
\cite{Zaimag96} as proposed by \texttt{CASPER} \cite{chien00}.

This last lesson in particular had a lasting impact with the
observation that planning and execution are strongly
inter-related. This core concept was behind the next generation of the
Remote Agent, called \texttt{IDEA} (Intelligent Deployable Execution
Agents) where planning and execution were \emph{intertwined} within a
single domain model that spanned the most abstract (for high-level
planning) to the least (for diagnosis). \texttt{IDEA} \cite{mus02,
  mus04} agents were expected to interact to achieve plan formulation;
however there was no systematic framework for formally governing these
interactions. \texttt{IDEA} was also computationally heavy and
required substantial effort to customize for an application domain. It
did move to a retooled version of \texttt{HSTS}, now called EUROPA
\cite{frank2003, barreiro09} and ported to the C++ language.

While \texttt{IDEA} was being deployed as a coordinating executive for
earth analog rover deployments \cite{wetter05} a separate development
was undertaken to deploy the EUROPA planner for NASA’s Mars
Exploration Rovers (MER) mission. The \texttt{MAPGEN}
(Mixed-initiative Activity Plan GENerator) planner \cite{bresina03,
  aichang04, bresina05, bresina05a} as a decision support tool in the
mixed-initiative mode; \texttt{MAPGEN} continues to be used to this
day to command the twin rovers on the Red
Planet. % While EUROPA2 was not
% designed as an embedded planner, NASA undertook a bottom-up
% re-implementation of the planner and substantially increased its
% performance. EUROPA2 was then released as open source to the AI
% community at large [Europa].

The Autonomous Sciencecraft Experiment (\texttt{ASE}) \cite{chien99,
  chien03} was another autonomy demonstration in space, this time for
a more observable vehicle, the EO-1 in Earth orbit. It encapsulated a
classical 3-layered architecture like \texttt{RAX}. The fundamental
functional difference was that goals could not only be sent by a
ground segment, but also by an onboard science driver which could
encapsulate a range of Machine Learned feature detectors which could
trigger the planner. However ASE was not an integrated system; the
\texttt{CASPER} planner \cite{chien00} would generate plans which were
executed by a separate commercially available rule-based system
\cite{icl}. \texttt{CASPER} touts a \emph{continuous} paradigm for
planning; however this incremental modification simply takes the
overall mission plan and selectively plans near-term at a more
detailed level. The system also uses iterative-repair as a way to fix
plans; however this too is derived from a model of the mission and
spacecraft constraints which allow simple reconfiguration modes
towards replanning. Robust software engineering is also not an
important aspect targeted by \texttt{ASE} given the disparate models
between the planner and the rule-based executive. With such a system,
real-time updates from the environment cannot be accommodated since
the planner is a monolithic computational environment. The rate of
updates from the science drivers is roughly \kcomment{on} the order of
an earth orbit on the EO-1 vehicle. As a result, the ASE is only an
incremental update on RAX and is not suitable for domains where
complex automated reasoning and asynchronous environmental
observations need to be factored in deliberation.

\texttt{TCA} \cite{simmons94} is another framework for robot control
and has similar motivations to \texttt{RAX} and
\texttt{ASE}. Reactivity and deliberation are also key to \texttt{TCA}
and a systematic approach to development led to its design. However it
suffers from three key weaknesses. First it has a weaker
representational framework of Task Trees which require representation
of tasks within cleanly formulated hierarchies. Constraints between
branches within a task hierarchy are allowed; timing between leaves of
different trees is however not possible. Second, its temporal
framework does not deal with flexibility; time points are fixed and
while partial orders are possible, it does deal with execution time
uncertainty. Third, while \texttt{TCA} shares the notion of using
domain expertise of different planners, it does so thru a centralized
message passing mechanism. While this allows disparate code-bases to
talk to each other \kcomment{through} IPC (\kcomment{an} inter-process
communication or message passing mechanism), failure in the
centralized controller is tantamount to a system crash.

The \texttt{ReSSAC} project at ONERA \cite{teichteil07} has similar
motivations in controlling aerial UAV platforms for autonomous Search
and Rescue. Like \texttt{TCA}, the project uses a supervisor to
control planning and at the same time decouples the deliberation and
reactive components from the lower level functional layer. This
deliberate separation is partly because of the use of optimizing MDP’s
(Markov Decision Processes \cite{mdp93}) for planning which uses
probabilistic state transitions to determine the next course of action
based on perceived state. To overcome the typical problem of state
space explosion with MDPs, a local heuristic is used in order to
generate reachable goal states incrementally. This work however does
not deal with metric time with a modestly simple model.

\texttt{CIRCA} \cite{musliner95} is another effort to bring decision
making for real-world problems with the augmented need to have
\emph{verifiable} controllers synthesized in-situ to ensure that
within explored states, there are no erroneous transitions to
dangerous states by asynchronously generating Test-Action-Pairs
(TAPs).  These are annotated production rules consisting of a set of
tests (or preconditions) and a set of actions to take if all the tests
return true. TAPs are synthesized and scheduled by \texttt{CIRCA} and
provide a viable way to deal with a dynamically changing
world. \texttt{CIRCA} however differentiates between reasoning about
time and reasoning in real-time with the implication that reasoning
necessarily requires substantial computation for state space
exploration which negates the real-time (and thus real-world)
impact. 

A key and early concern that dominated planning was that of
scalability.  The planning cycle in many approaches was monolithic
often making fast reaction times impractical when necessary for
embedded robotic agents. Many of these systems (which were
$3$-layered) also utilized very different techniques for specifying
each layer in the architecture resulting in duplication of effort and
a diffusion of knowledge.  The work we highlight in this chapter
builds on the approach used by \texttt{IDEA} \cite{mus02, mus04} in
utilizing a collection of controllers, each interleaving planning and
execution within a common framework. \texttt{IDEA} however, provided
no support for conflict resolution between controllers, nor does it
provide an efficient algorithm for integrating current state within a
controller, relying instead on a possibly exponential planning
algorithm. Efficient synchronization of state in a partitioned
structure is fundamental to making the approach effective in practice.

Our system, \rxe, the focus of this chapter, is an agent control
structure formally defined as a composition of coordinated control
loops where sensing, planning and acting are resulting from concurrent
deliberating tasks within a formal framework.  \rx was designed after
carefully evaluating lessons learned from \texttt{IDEA} to which it
owes substantially the notion of partitioned problem solving. However
partitioning in \rx is systematic and methodologically principled
where we manage the information flow within the partitioned structure
to ensure consistency in order to direct the flow of goals and
observations in a timely manner. The resulting control structure
improves scalability since many details of each controller can be
encapsulated within a single control loop.  Furthermore, partitioning
increases robustness since controller failure can be localized to
enable graceful system degradation, making this an effective
\emph{divide-and-conquer} approach to the overall control problem.

A recent variation of the key idea of such controller composition
derived from \rx is \texttt{ROAR} \cite{degroote11} where lower-level
functional modules that control specific sensors are organized within
a graph hierarchy. One reason they cite such a structure is essential
is in dealing robustly to failure; an off-nominal condition will allow
rapid graph traversal to identify alternative ways in which sensing
tasks can be executed and therefore aiding in responsiveness. However,
it is unclear whether to date this system actually deliberates and
whether it is actually instantiated on a real platform for control.

% ``The whole partition must be accordingly reorganized: this kind of
% construction does not scale well over a large variety of missions,
% missing the objective of a versatile architecture for robots.''

%%% Local Variables: 
%%% mode: latex
%%% TeX-master: "setobook"
%%% End: 


\section{Planning and execution in the \rx architecture}
\label{sec:arch}

% {\em\tiny This section introduces shortly T-REX architecture but move
%   quickly the focus on a single reactor and gross concepts on the
%   architecture:
%   \begin{itemize}
%   \item For t-trex the whole decision and control problem is reduced
%     around the state variables (or timeline construct). 
%   \item This reduce the execution tracking problem for each reactor on a state
%     identification problem (deduce internal state from the external
%     state), and the control problem to goal posting which will trigger
%     deliberation on the owner of the corresponding timeline. 
%   \item the architecture itself see each reactor as a black box and
%     therefore each reactor can implement its own mechanism in order to
%     resolve both synchronization and deliberation. Still we provide a
%     reactor based on the europa framework that leverage the automated
%     planning capabilities in order to do model based planning and
%     execution with a rich representation of resources and time.
%   \end{itemize}}

\subsection{Introductory concepts}
\label{sec:arch:intro}

A prime motivation behind the \rx architecture was to design a
controller which could bring automated planning closer to low-level
control of the vehicle while maintaining its reactivity. Traditionally
planning has been deemed computationally expensive with planner
performance restricting robot reactivity
\cite{ghallab04,Dias:2003ua}. In control architectures where task
planning is embedded, planning consequently remains at an abstract
level with the assumption that it would otherwise impede system
reactiveness and that the environment can change at a faster rate than
the planner can plan for it. In such a situation, the agent may thrash
if the internal state of the plan gets out of synch with the actual
state of the world. As a result the planning problem managed {\em in
  situ} remains fairly detached from low-level control. Most
adaptations at the lower-level reactive layers in contrast are managed
by specialized components (for example GeNoM \comment{} and CLARATy
\comment{}) that rely on different representations for modeling local
control behavior or worse do not have any explicit formal agent
behavior model. The consequent different techniques for specifying
each layer in the architecture results in duplication of effort and a
diffusion of knowledge leading to significant design and integration
issues \cite{DS1report}.  \texttt{IDEA} mitigated most of these issues
by tightly integrating planning and execution in a single
representational and computational framework with a unified
declarative model. Our approach with \texttt{T-REX} builds on this
design methodology albeit in a substantially systematic manner

The \rx architecture contributes to agent architectures over and
beyond \texttt{IDEA} are in the following ways:

\begin{itemize}

\item \emph{Partitioning} the global planning problem into multiple
  decision loops, called {\em reactors}. Each reactor has its own
  scope both functionally (one for example focusing on the conversion
  of high level waypoints into low-level commands) and temporally as
  each reactor declares its planning look-ahead and expected latency
  before being able to produce plan to the agent depending on where it
  is located in a reactor dependency graph such as shown in
  Fig. \ref{fig:agent}.

\item \emph{Coupling} a tighter integration between the planning
  process and world evolution to ensure that planning occurs while
  being continuously informed of state change. By doing so, the
  planner is better informed in its deliberation despite a larger
  latency than changing world change state.

\end{itemize}

\rx provides a formal framework which details partitioning in
\cite{py10} and \cite{rajan12} and is not the focus of this
chapter. Coupled interaction between planning and the world on the
other hand, has a significant implication on how planning can be
integrated in such frameworks; here we show using the \eu planner can
be coupled to our framework. First we link concepts in \eu planning
with the \rx framework. Note that in the following treatise, we call
the formal semantic framework of \rx as the 'agent'. Reactors
constitute an entity within this agent and are standalone blocks of
inference.

% on should integrate a
% planner in this framework in general and was applied to the europa
% framework specifically. We'll develop later how this requirement
% implies to see in-situ planning as two parallels processes that are
% tightly linked through the same plan structure. But before that we
% need to introduce at least the high leve lconcepts and ideas behind
% our architecture as a whole.

\subsection{Architectural concepts in \rx}
\label{sec:arch:trex}

The \rx architecture is structured around the notion of \eu {\em state
  variables} as the fundamental basis of interaction. Additionally it
comes with a well-defined ownership model associated with each state
variable. In this framework each reactor is seen as a 'black box'
providing its own {\em internal} state variables which it is
responsible for in order to maintain its value at every instant of
time. Each such instance is represented by a {\em tick} with a fixed
duration; in our implementation of \rx on the Dorado, we use a tick of
$1$ second. In order to identify its state a reactor can subscribe to
{\em external} state variables owned by other reactors, for which it
will receive new updates when the owner reactor generates its {\em
  internal} state.

This {\em synchronization} process propagates up through the reactor
dependency graph bottom up at every {\em tick} ensuring that all the
reactors have a consistent view of the world. Conversely each reactor
can request a future change on one of its {\em external} state
variables which will be transfromed as an {\em internal} goal to the
owner. Such \emph{goal dispatching} is expected to produce a
deliberation phase for the receiving reactor which can cyclically and
in turn, produce a plan impacting its {\em external} state variables
that will propagate down following the same mechanism.

% Taking cues from the IDEA 
% architecture \cite{mus02}, we designed a core framework that provides a
% formal basis \cite{Py:2010ti} on the way each of the sub-component --
% called reactors -- of our agent can interact by exchanging only
% information through {\em state variables} by ensuring that state 
% information can propagate through all the reactors ensuring a
% consistent view of the present state of the world at every single tick
% trough a bottom-up {\em synchronization} flow and collaboration for
% future state evolution by timely exchanges of goals on {\em state
%   variables}. This exchange does not impose to any reactor to be aware
% of the reactors it is connected to but only focus on the state
% variables it relies on (called {\em external}) and the state
% variables it manages and maintains (called {\em internal}). 


\begin{figure}[!htb]
 \centering
 \includegraphics[scale=0.45]{figs/AUV-agent.pdf}
 \caption{\small A \rx agent is composed of multiple reactors or
   control loops (rounded boxes) which are connected through state
   variables provided by one reactor ({\color{blue}blue} solid line)
   with multiple possible clients ({\color{green}green} dashed lines)}
  \label{fig:agent}
\end{figure}

Consider for example the \rx agent instance we use on our AUV as shown
in Fig. \ref{fig:agent}. In this instance all the reactors (symbolized
by colored boxes) do not have an explicit connection from one to
another but instead rely on a publish/subscribe model on shared state
variables. For instance the \textsf{Pilot} reactor needs the {\em
  externally} managed state information from \textsf{Sensor\_Data} and
\textsf{Nav\_cmd} state variables in order to maintain the
\emph{internal} state variables \textsf{Waypoint} and
\textsf{Nav\_policy}. This relation applies in two ways:

\begin{itemize}

\item In order to identify its current {\em internal} state values the
  \textsf{Pilot} \comment{double check this mod} needs to know the
  current {\em external} state information it relies on. In this case
  the current \textsf{Waypoint} the vehicle is heading to can be
  extracted for example from the current \textsf{Nav\_cmd} executed.

\item The future objectives a reactor has on \textsf{Internal} state
  will likely imply sub-objectives to be reported to the owner of its
  {\em external} state variables depending on their current
  values. Should the \textsf{Pilot} want to visit a specific
  \textsf{Waypoint} -- given its current position as provided by
  \textsf{Sensor\_Data}, it can identify a sequence of
  \textsf{Nav\_cmd} states that should eventually help it reach this
  location. \comment{this explanation could do with a figure which I
    believe we have used previously}

\end{itemize}

Given the above example, the \rx architecture therefore helps abstract
reactor interaction by containing them to state variables. Each
reactor is therefore agnostic where its {\em external} state
information is managed as long as this information is available and
properly maintained for managing {\em internal} state
information. This {\em internal} state in turn may be used by other
reactors. Such a formalism also constrains reactor interaction to
state information exchange which comes in two forms:

\begin{itemize}

\item {\em Observation on current state values}: This information
  propagates \emph{up} in the reactor dependency graph at the
  synchronization phase and occurs at the beginning of every
  tick. During this phase the agent provides updates to each reactor
  on their {\em external} state variables for this tick so as they
  could compute their {\em internal} state variable values for this
  same tick. By doing so we propagate a consistent view of that state
  of the world throughout all the reactors as time
  advances. \comment{could do with a figure here}

\item {\em Goal request on future state}: This information propagates
  \emph{down} in the reactor dependency graph. In order to satisfy own
  {\em internal} objectives, a reactor may have a plan that relies on
  future values of one of its {\em external} state. When such a plan
  is identified the agent ensures that this part of the plan is given
  as a goal to the owner of the given {\em external} state
  variable(s). \comment{could do with a figure here}

\end{itemize}

The choice of representation of state variables, observations and
goals is directly derived from \eu timelines and tokens.  While such a
rich representation allows exchange of state information in a formal
yet flexible manner, its translation into planning frameworks that
have an explicit representation of time (such as but not limited to
\eu) is often trivial.

\subsection{The Execution Cycle}
\label{sec:arch:exec}

The overall execution cycle of the agent is also abstracted out as a
continuous planning/deliberation cycle between all reactors. At the
agent interface level, each reactor provides only two abstract calls
that conceptualize execution:

\begin{itemize}

\item \texttt{synchronize} takes the last {\em external} observations
  as an argument and returns the {\em internal} observations for a
  reactor or a failure report.

\item \texttt{step} that accepts new goals requested (if any) on the
  {\em internal} timeline of a reactor and will execute one step of
  deliberation \comment{Might be good to articulate what ``one'' step
    looks like}. The returned value indicates if a complete plan has
  been found. If so this part \comment{what does 'this part' mean? Can
    we sketch a figure or make it more formal?} of the plan is
  provided to {\em external} timelines of this reactor.

\end{itemize}

Both these deliberative functions have different scope and temporal
constraints. The \texttt{synchronize} call is considered as atomic and
will focus only on state inference for the current tick by identifying
the current value of {\em internal} state variables from the latest
updates on the {\em external} state variables. This process propagates
up in the reactor dependency graph at any single tick \comment{at
  every single tick??? maybe} to ensure that all reactors are aware of
the current state of the world. \texttt{step} relates to a single
decision step of the planning process which is focused on the future
evolution of state variables; it can take multiple steps spanning
multiple ticks for a reactor to produce a plan that is both complete
and valid. When such plan is identified \comment{how? Is there a
  terminating condition?} its {\em external} timeline(s) are then
dispatched to the corresponding reactors which own these times and
which in turn can start their own deliberation \texttt{step}s to
satisfy these new objectives. Consequently execution occurs as these
goals and plans propagate down in the reactor hierarchy until they
eventually become goals for the lowest level reactor which -- instead
of deliberating -- may just transform this goal into a command for
execution in hardware.

\begin{figure}[!htbp]
  \centering
  \vskip-1pc
  \includegraphics[width=0.55\columnwidth]{figs/tick-cycle}
  \caption{\small \texttt{T-REX} reactor execution cycle: {\em
      Deliberation} is interrupted by {\em synchronization} at the
    beginning of every {\em tick} allowing integration of state
    information.}
  \label{fig:tick-exec}
  \vskip-0.8pc
\end{figure}

The manner in which the agent interleaves planning \texttt{step}s and
synchronization for a single reactor is illustrated in
Fig.~\ref{fig:tick-exec}. Synchronization occurs at every tick
interrupting planning; this allows a reactor to identify its current
state that can propagate through the reactor hierarchy in order to
ensure a consistent view of the state of the world between all the
reactors.
% This view of a reactor has two parallel tasks allowing for long
% planning deliberation while ensuring that the state evolution
% propagates through the agent as time advance.
These intertwined tasks together provide two natural information flow
methods (bottom-up for state estimation and top-down for plan
projection) that we expect to see in a control loop. As they are both
inference-based processes it is natural to implement a reactor based
on generic automated planning. For example in Fig. \ref{fig:agent},
apart from \textsf{Shore comm.} and \textsf{Vehicle} acting as
interfaces to the system, every other reactor (in blue) are different
instances of a \eu reactor differentiated only by their model and \eu
solver configurations.

\subsection{The  \eu Deliberative Reactor}
\label{sec:arch:europa}

The \eu reactor is our implementation inside \rx of a reactor that
usess automated planning for its execution. This reactor is fully
based on the europa framework and applies the core idea of \rx we
described above to implement a reactor for which the sole purpose is
to deliberate as time advance. The design of this reactor takes cues
from the IDEA architecture presented on \cite{mus02, mus06} in the
sense that both planning and execution are based on a single unified
model. The choice of \eu as the planning framework was directed by the
fact that this framework provide a rich and highly configurable
toolset to be able to infer around a rich temporal domain.

This reactor is implemented following the idea that in \rx, the
whole execution cycle is two deliberation processes. Both of these
processes are implemented as two europa solvers modifying the same
plan database and for which the execution is managed by the \rx 
framework as time advance during the execution of the system:

\begin{enumerate}

\item The synchronization solver is a specialized europa solver that
  will integrate in the plan new information about the evolution of
  the External state variable and ensure that the reactor propagate
  these in order to both identify its current state and inform
  other reactors of any state change on its internal timelines. This
  solver is summoned at the beginning of every single tick.

\item The deliberation solver is managing the deliberation process of
  the reactor in order to produce a new plan or alter its current plan
  as new goals are given to the reactor or the synchronization solver
  identifying a conflict between the current state of the world and
  the expectations of the previous plan. This process can  span other
  multiple tick and therefore can be interrupted at any single tick in
  order for the synchronization solver to do its task.

\end{enumerate}

While the two processes are separate, they share the same plan
internal to a reactor and therefore at every synchronization cycle the
planning process is informed of the new world state and its impact on
the plan. Conversely, when the planning process eventually finds a
solution, synchronization is informed about generating the planned
state values on the {\em external} state variables. This {\em
  external} plan defines the set of {\em goals} for this state
variable managed by another {\em reactor}.

An important challenge for embedding automayted planning i na situated
agent as always been to go around the asumptions made by most classical
planners. Specifically one assumption restrict the problem to ``offline
planning''  defined in \cite{ghallab04} as follow:

\begin{quotation}
  The planner is not concerned with any change that may occur in
  $\Sigma$ \footnote{In this book $\Sigma$ identifies the world modeled
    by the plan domain} while it is planning; it plans for the given
  initial and goal states regardless of the current dynamics, if any.
\end{quotation}

This assumption, while helpfull to reduce the scope of the planning
problem, becomes problematic if one wants to embed a planner in a
real-time system that evolves in a highly dynamic environment. In a
typical robot this implies that the only point where the planner can
be informed about the current world state is right before the planning
is started by giving to the planner the intial state and current
objectives. During the planning, it is assumed that the agent can
maintin the world as perceived by the planner as stable. In
\cite{lemai04, lemai-chenevier2004}, they do attempt to reduce the
impact of planning by allowing local plan repairs which attempt to
insert repairing actions in the exisiting plan andf can be done fast
enough. Still in ythe eventuality that such repair is not possible
they add to stop their vehicle until the full (re-)planning was
complete. {\em I may need more examples/refs here}

Looking back at Fig. \ref{fig:tick-exec}, one can already see that
this asumtpion is problematic in our architecture. Indeed
synchronization that tracks state evolution can occur several times
during the reactor deliberation process. By making synchronization and
deliberation share the same plan we provide a solution that relax this
assumption. In the following sections we present how synchronization
and deliberation are implemented independently first to finally show
how their interaction through the plan helps relax this assumption by
making the planning fully integrated to the reactor execution cycle
while potentially allowing for more informed planning.

\subsubsection{Synchronization identify internal state evolution}
\label{sec:arch:synch}

For the sake of simplicity we introduce first how a reactor can track
and identify its state in the context where it does not have any
compelling need to deliberate. Assume here that we have a reactor that
have no future goal. {\em Need to develop the reason why it is a
  necessity: primarily to provide its internal state to whoever
  observe it but als simply to ensure that its current representation
  of the world is up to date and still consistent}

In the europa based reactor we introduce this requirement as a new
type of flaw for the europa framework. This flaw enforces at the
reactor to identify fully its internal state for the current tick. By
using this new flaw we can describe the synchronization process as
this general sequence :

\begin{enumerate}

\item Integrate the external state as provided by the owner of each
  external timeline into the plan database.

\item Propagate this information in the plan database following the
  model $\mathcal{M}$ of this reactor.

\item Resolve the current state value of each internal timelines.

\end{enumerate}

\begin{figure}[!htbp]
  \centering
  \includegraphics[width=0.5\columnwidth]{figs/synch-relation}
  \caption{Illustration of the synchronization flaws in a reactor. The
    reactor receive new observations when they are produced by the
    owner(s) of its internal timelines. The line after the last token
    of each timeline represent the domain of possible values for the
    end of this token. At every tick $\tau$ the reactor needs to
    integrate the {\em External} state information it received and --
    based on its model $\mathcal{M}$ -- resolve its {\em Internal}
    state that will then be provided by the architecture to other
    reactors using these state variables.}
  \label{fig:synch:flaw}
\end{figure}

The resolution of internal state flaw can be resolved using one of the
following choices which are evaluated in the given order:

\begin{enumerate}

\item Extend the previous state value to end after this tick (ie
  restrict its end time to $[\tau+1, \infty)$). 

\item Start the next active token in the timeline and
  attempt to start it now (ie restrict its start time to the single
  value $\tau$).

\item Create and insert a new token in this timeline that will start
  at the current tick $\tau$ (attempt this for each possible token
  type for this timeline if necessary). 

\end{enumerate}

All of these choices are evaluated sequentially until a consistent
solution with no more flaw for this tick is identified. In order to do
so we need to make the assumption that the current state value of an
internal timeline do not depend on the future or more accurately that
any choices made during this synchronization will not lead to a future
inconsistency (meaning no possible solution) for a future
synchronization. Such assumption deeply impact the set of possible
domains one the reactor can support while remaining complete. Take for
example this rule from our \texttt{Shopping} model :
\begin{verbatim}
 1 Agent::Go {
 2   met_by(condition object.location.At origin);
 3   eq(from, origin.loc);
 4
 5   equals(effect object.location.Going going);
 6   eq(going.from, from);
 7   eq(going.to, to);
 8   
 9   meets(effect object.location.At destination);
10   eq(to, destination.loc);
11 }
\end{verbatim}
While this model is perfectly standard and acceptable in the overall
case. It can become problematic while tied to the
execution. Specificially it puts a strong tie between the {\em fact} that we
are \texttt{Going} to a location and the {\em expected} future outcome  of
ending at this location. This ties is even stronger due to the
constraint at line 5. Knwoing this aspect assume that the
\texttt{Agent.location} is {\em external} to the reactor which receive
the observation \texttt{Going(Home,SuperMarket)} which synchronization
resolved by producing the observation \texttt{Go(Home, SuperMarket)}
in the Agent timeline. From this point this means that the only
possible outcome on \texttt{Agent.location} is to be
\texttt{At(SuperMarket)} in the directly foreseeable future. In the
real world the fact that we are \texttt{Going} to a certain location
does not necessarly imply that we will succeed on ending up there. An
extreme case could be that due to some exogenous event, the road to
the \texttt{SuperMarket} is closed which could result on an alternate
observation \texttt{At(7th st. \& Foo Ave.)}. This observation
contradicts thre existing plan the reactor maintin and it is important
here to relaize the temp[oral direction of the causal link that is
triggereing this inconsitency. Indeed at the time we will observe this
contradicting \texttt{At}, the rule that generated this conlict is
connected to the past observations \texttt{Going} and \texttt{Go}
which cannot be fully retracted without potential global impact in the
rest of the exisiting reactors -- implying  a huge cost in
backprogation that would be problematic. 

More generally tying a token
to future outcomes in a model that is meant to be confronted to rela
executionis something that is problematic. It may be valid ifg you can
gurantee that this will always happen (for example the reactor
managing \texttt{Agent.location} timeline guarantees that the outcome
of the \texttt{Going} will {\em always} result on ending at the traget
location) but should be manipulated while keeping in mind that
predicting the future is not absolute. In this case one could alter
this part of the model as follow in order to give the reactor more
flexibility:
\begin{verbatim}
 1 Agent::Go {
 2   met_by(condition object.location.At origin);
 3   eq(from, origin.loc);
 4
 5   contains(effect object.location.Going going);
 6   eq(going.from, from);
 7   eq(going.to, to);
 8   
 9   meets(effect object.location.At destination);
10   eq(to, destination.loc);
11   going meets destination;
12 }
\end{verbatim}
The alteration made to relax the constraint on linbe 5 from a strict 
\texttt{equals} to contains and then add the new constraint at line 11
that ensure that the last \texttt{Going} done during this \texttt{Go}
ends up in the location the agent wnats to be. This firts level of
relaxation will therefore allow the agent to usies multiple
\texttt{Going} should the first one fail to reach our objective.  The
probalem can even be relaxed further if needed but this relaxation
gicves already more felxibility for the reactor to recover from local
failures.

Synchronization remains here a deliberation process, and for this
reason there's always the risk that the search of a solution is not a
straightforward sequence of unit decision. As such process is required
by the architecture at every ticks for every reactors presently
running this need to be carefully designed. At the engine level we
made careful choices in term of both the focus of the search --
limited only to tokens that can potentially overlap the current tick
-- and its heuristic where the internal state flaw is considered last
during the search. The later proved to limit the amount of backtrack
in the search for a solution -- or at least how deep we need to
backtrack in our decision tree -- similarly the sequence in which we
evaluate the possible resolution for such a flaw -- presented
previously -- was selected according to simple assumptions that takes
benefit of a possibly already existing plan by assuming that if the
outcome of synchronization appears consistent with our current plan
then it probably reflects that this reactor plan is executed. Such
assumption helps direct the internal state evaluation and, in nominal
cases (where we assume that reactors are acting in concert), proved to
be a sound solution.

One final aspect to keep into account is that in \rx point of view
synchronization is the only critical task of a reactor. Indeed, such
process is used in order for the agent to identify its current state
and maintain it consistent throughout the agent life-time. A reactor
not being able to find a consistent solution during synchronization --
meaning that after evaluating all the possible solutions provided by
its model it did not found any plan that is not inconsistent -- not
only expresses that this reactor is not able anymore to explain the
state of the world but may jeopardize other reactors that relies on
its {\em internal} state. For this reason a failure to do
synchronization immediately result on the agent killing this reactor
and notifying any reactor that depends on its {\em internal} state
variables by posting the special observation \textsf{Failed}. By doing
so we allow for graceful degradation where other reactors can readapt
their plan as they receive this \textsf{Failed} observation.

\subsubsection{Deliberation : planning for future state evolution}
\label{sec:arch:plan}

The planning process inside a reactor is directly based on the way
europa work which was described in previous section. Still some simple
alterations were made in order to both include deliberation
information one reactor provide to the architecture -- such as its
deliberation latency and its planning look-ahead both expressed in
ticks -- and the fact that we potentially need to interleave this
deliberation with not the synchronization process but also with the
deliberation of other reactors\footnote{Current implementation of \rx
  is running on a single process and it is the responsibility of the
  agent itself to emulate the multi-threading of the reactors
  deliberation and synchronization}. For this reason while the overal
planning of a europa reactor strictly relies on the europa framework
plnanning solver, its slices the execution of Algorithm
\ref{alg:europa:solve} in atomic steps. The interruption is done
around the recursive call (line \ref{li:recurse}) while always
ensuring that the current partial plan mainitained is never proven
inconsistant at any single step. This can be managed by allowing the
call to backtrack until a DecisionPoInt that is not exhausted (ie
$decision.hasNext()$ is \texttt{true}) is find in the call stack.

One critical aspect of this deliberation step is to ensure that at the
exit of this call the $plan$ is not proven inconsistent. Indeed, it is
always possible that the next call made by the agent is the
synchronization which would immediately fail if the $plan$ is
inconsistent before any solving attempt made for synchronization.  For
this reason if the plan is proven inconsistent during a deliberation
step we either backtrack in the decision stack until we find a node fo
the tree that have remaining alternate solution or simply fully relax
the plan in situation there's absolutely no other alternative (by
removing all the goals and keeping only recent observations
produced). These constraints are yet again related to the fact that
for the agent the only crucial functionality of a reactor is to
resolve successfully its synchronization.

The number of flaws between two steps can evolve for the following
reasons that are exogenous to deliberation :

\begin{enumerate}

\item A new goal has been posted on one of the reactor timeline. This
  is usually due to another reactor finding a plan and posting the
  corresponding new objective to this reactor.

\item The outcome of synchronization created new flaws that did not
  need to be resolved during synchronization.

\item A previously requested goal has be recalled by the initial
  requester. This can happen when the reactor did identify that is
  initial plan is no longer valid. For example as synchronization
  proved this plan to be inconsistent.

\end{enumerate}

These 3 events will potentially produce new flaws to be resolved while
any single deliberation step will usually reduce the number of flaws
to be resolved.

The scope of planning for the reactor plays also a role in the
deliberation process as we limit deliberation to the sliding temporal
window $[\tau+\lambda, \tau+\lambda+\pi]$ where:

\begin{itemize}

\item $\tau$ is the current tick

\item $\lambda$ is the specified latency of this reactor. Which is an
  indicator on the expected maximum time this reactor is expected
  before being able to produce a plan.\footnote{Do note that this
    parameter remains indicative and, while it is highly recommended
    to select a sound value, a failure to produce a plan in this delay
    is not considered as a critical failure of the reactor}

\item $\pi$ is the planning look-ahead of the reactor. which indicates
  how many ticks in the future this reactor is looking ahead while planning.

\end{itemize}

This scope allow to filter out any tokens that are either necessarily
ending before $\tau$ or necessarily starting after. Focusing the
[planning problem to only the tokens that are in a reasonable
future. This window is taken into account by the agent which will
notify the reactor of new {\em internal} goals only when they overlap
this window but also at the deliberation solver in order to reduce the
number of tokens to be evaluated and consequently reduce the cost of
deliberation. As time advance flaws that where initially beyond this
window will become active flaw that the reactor can then evaluate
resulting on an apparent continuous planning of the reactor.

When there are no more flaw is present in the current planning scope a
plan solution is found. This has two effects:

\begin{enumerate}

\item This reactor does not need to deliberate until the next synchronization.

\item The {\em external} part of this plan can then be posted which
  eventually (at the end of the current tick if these goals overlap
  the owner planning window) will be dispatched as goals triggering
  new deliberation on these reactors.

\end{enumerate}

\subsubsection{Intertwining synchronization and execution}
\label{sec:arch:intertwine}

A core aspect of this reactor is that it takes advantage of the
specific sequencing enforced by the \rx architecture along with the
fact that only one $plan$ structure is maintained in the reactor and
shared between synchronization and planning to allow these two
processes to not only be interleaved but impact each other through the
way they manipulate this plan.

The most obvious case is how synchronization allow propagate of {\em
  external} observations as a new tick occur in the plan. In or
presentation of synchronization, we isolated the problem by stating
that we will assume at this point that the reactorhas no future goal
or more accurateely no plan to enforce at this stage. By doing so we
were able to develop the way we augment yhe europpa solver to be able
to propagete {\em external} observations in order to identify the
reactor {\em internal} state for the same tick. By doing so we were
able to focus on synchronization as a state evluation process that can
be resolved within the europa framework with few extensions. simliarly
we presented deliberation at this fairly high level not necessarily
discussing how the disruptive nature of synchronization in this
process could impact the way this planning process will evolve.

We first anlyse the case during which synchronization occurs after the
planning has vcompleted (ie the last seep of deliberation resulted
into a plan that is considered as complete for the current
look-ahead). As synchronization starts with the plan produced this
will gave a frame for the search process. Therefore this deliberation
will not only be a model-based estimation uniquely based on the {\em
  external} observations but will also include the reactor intentions
as described by the plan produced. Consider for example the Shopping
Agent model described during the europa presentation and that the
location of the \texttt{agent} is an {\em external} timeline. For
example as, the reactor add the goal to have \texttt{Own} milk and the
current {\em external} states indicated that it was \texttt{At(home)}
it produced the plan in Fig. \ref{fig:shop:exec0}. The {\em external}
part of the plan -- namely the tokens \texttt{Going(home,
  superMarket)} and \texttt{At(SuperMarket)} -- and integrated as
goals for the reactor managing this specific state varaiable {\em
  internally}.

\begin{figure}[!htb]
  \centering
  \includegraphics[width=0.7\columnwidth]{figs/shoping_exec_t0}
  \caption{Shopping example : plan produced before synchronization. 
    Tokens in red indicate observations, 
    tokens in blue the goal that this plan attempt to solve. Arrows
    indicate temporal constraints.}
  \label{fig:shop:exec0}
\end{figure}

Now, consider that at the next tick $\tau+1$ the reactor managing the
agent location was able to change the location state varaibale to the
\texttt{Going}. \rx informs our reactor of this new observation which
is added in the plan structure as a fact -- starting exactly at
$\tau+1$ and lastign for an unlknown duration which is at least 1
tick. As this new observation is integrated in the plan through
synchronization the solver can identify the similarity between this
new observation oand the next token in the existing plan. Which mean
that instead of making a blind exhaustive search of all the possible
implication the synchronization process starts its execution with the
assumption that these new observations are consitent with the current
plan maintained. As a result at the end of synchronization we will
have the plan depicted in Fig. \ref{fig:shop:exec1}. Which just
propagated the information provided by the new observation in the
exisiting plan (restricting in turn the start time of \texttt{Going}
and by propagtion of the constraints \texttt{Go} to be {\em exactly}
$\tau+1$). While focusing uniquely on the execution frontier ($\tau+1$
in Fig. \ref{fig:shop:exec1}), the europa solver took benefit of the
exisiting plan to evaluate a solution that asumes that the observation
is consistent with what was decided inside this reactor.


This choice is directed not only by some natural heurisitc choices
made in the solver but even more enforced in general by the heuristic
order of the decision point reolution choices for our {\em internla}
state flaw presented in section \ref{sec:arch:sync} {\em\color{red}
  TODO: need to rework the presentation of thee choices so I can refer
  to them more directly -- algorithm or pseudo theorem should do the
  trick}. Should such a flaw be revealed during synchronization --
meaning that the current state of an {\em internal} state varaibale is
not fully grounded -- the sequencing of the choices will atempt first
a conservative choice in term of maintaining previous state, then an
optimisitic choice in term of the execution/advance in our current
plan to finally attempt other choices that were not predicted by the
model. The two first possible choices are strongly directed by the
exisiting plan which help direct the search within this frame. This
often allow to have a synchronization that is much more directed and
i, in nominal situation such as the one we showed resolve the
deliberation in few steps (often without any backtrack in the search).

\begin{figure}[!htb]
  \centering
  \includegraphics[width=0.7\columnwidth]{figs/shoping_exec_t1}
  \caption{Shopping example : Result of synchronization after the {\em
      Going} observation was received by the reactor with thew plan
    showed in Fig. \ref{fig:shop:exec0}.}
  \label{fig:shop:exec1}
\end{figure}

The synchronization process is also in this process supporting fully
the tracking of the plan execution as it specifies tokens parameters
such as the time points and allow this to propagate through the plan
in order to idenitfy if the plan remains valid -- as in our example --
or that it cannot be executed as it is. The later is often identified
by the synchronization failing to find a consistent solution. for
example dshould the \texttt{Going} observation we receive be to the
\texttt{hardwareStore}, this would break the current plan of our
reactor as we cannot go simultaneously at 2 different places as
illustrated in Fig. \ref{fig:shop:relax}-1. as synchronization is not
possible in this context our strategy is to fully relax all decisions
made by the previous plan steps by just keeping n the plan all the
past observations in both {\em external} and {\em internal} state
variables and the exsiting {\em internal} goals received by this
reactor (\texttt{Own(Milk)} in our example) resulting on the plrtial
plan sghowned in Fig. \ref{fig:shop:relax}-2. In this process the
reactor inform the agent that all {\em external} goals it requested
for the future are note valid anymore which often result in the ownere
of htese timelines to not have to maintain this goal anynmore.

\begin{figure}[!htbp]
  \centering
  \includegraphics[width=0.7\columnwidth]{figs/shoping_exec_relax}  
  \caption{Illustration of a conflict suring synchronization and the
    resulting relaxed plan for recovery} 
  \label{fig:shop:relax}
\end{figure}

As the plan is not anymore in the way the synchronization can resume
and find a solution as we initial lly described. The newly pending
goals will enforce at the next \texttt{step} to resumed deliberation
on this new fully relaxed partial plan. By the simple use of
synchronization the reactor was not only able to identify that its
plan was not executable anymore but recover from it while offering a
new incomplete partial plan to be resolved during the next
deliberation \texttt{steps}. This on its own ogfer the core
functionality one should except to close the loop between planning and
execution in the sense that synchronization plays the role of what is
typically named as the {\em Executive} in other architectures
{\em\color{red} references to diverse archis, 3-tiered, laas, claraty,
  B Wiliams, and many more}.

In our case though the interaction between planning and execution does
not stop at this level. A core point of the overal schedulling of both
{\em synchronization} and {\em planning} steps resides on our
architecture enforsincing that the synchroniuzation process can occur
in between any step of the plan. This lead us to the fact that not
only synchronization is imnfluenced by the outcome of planning but
this influence goes also in the alternate direction. Specifically when
synchronuiation occuers before we found a complete plan, it will
inject as we described new facts (the {\em external} observations and
resulting {\em internal} state) in the plan database. This
informatiuon will be present when the next planning \texttt{step} will
resume. Therefore they are not intergal part of the plan resolution
(generating potentially new flaws for the planner or contributing to
resolve previously exisiting flaws). The planning process is therefore
perturbed internallyand fully informed on how the world evolves as it
is planning.  This aspect trueluy contrast to other approaches we have
identified in the literature which avoid to perturb the planner during
it search as it is not compatible with the ``off-line planning''
assumption.

By our design choices -- both at the architecture level and the
integration of the europa framework for embedded deliberation -- we
were able to implement an architecture that avoid this limitation
while remaining in a formal frame that still allow for deliberation
and model-based agent control. We believe that this tighter
integration of planning and execution allow the system to ba more
acute of its environment allowing better informaed decision and
tighter reaction. This is possible while still integrating information
such as a long latency for the reactors as long as one can ensure that
synchronization of al the reactors and deliberation steps can be done
in a time that is resonably small in comparison of the agent tick
duration.





% Gives a high-level overview of T-REX, the general design principles and how
% these principles aid in software engineering. Show T-REX block diagram.



%%% Local Variables: 
%%% mode: latex
%%% TeX-master: "setobook"
%%% End: 


\section{Foundational Concepts}
\label{sec:concepts}

\texttt{T-REX} is an adaptive, artificial intelligence based
controller and provides general framework for building reasoning
systems for real-world autonomous vehicles. At MBARI \texttt{T-REX} is
used for AUV control; another instantiation of the system is being
used for control of a terrestrial personal robot \cite{pr2,
  Meeussen:2010dn}. The development of \texttt{T-REX} has been
targeted at surveying a number of oceanographic features which are
dynamic and unpredictable spatio-temporally. Typically this requires
our AUVs to balance the goals of coverage spatially while
opportunistically following features of scientific interest and to do
so while being operationally aware of its own limitations in terms of
resources (typically the battery state of charge) and overall
proximity to other observational assets for obtaining scientific
ground-truth. We are therefore interested in representational
frameworks that allow robots to pursue long-term objectives while
choosing short-term gain and can gracefully deal with exogenous or
endogenous change.

To enable this responsiveness to external observations, the agent has
to be able to synthesize plans insitu and to re-plan. We use a
temporal constraint-based planner \eu with a demonstrated legacy of
having flown on NASA space missions \cite{jonsson00,bresina05,
  barreiro09}. Our autonomy architecture brings three key innovations
for AUV adaptation: the use of flexible plan representations,
compositional control with the use of partitioned networks and
off-line learning to inform insitu environmental state estimation. In
this section we detail some of the key concepts relevant to flexible
plan representation. Section \ref{sec:arch} has details the
\rx partitioned architecture and estimation driven vehicle
adaptation.

\subsection{The basics}
\label{sec:basics}

\eu uses a \emph{domain model} written in a declarative language
(NDDL: New Domain Description Language), together with initial
conditions and goals also specified in NDDL, to construct a set of
temporal relations that must be true at the start time. These models
include assertions about the physics of the vehicle, i.e how it
responds to external stimulus and internally driven goals. By
propagating these relations forward using Simple Temporal Networks
\cite{dechter91} and applying goal constraints, \eu can select a set
of conditions that should be true in the future, where some of these
conditions will correspond to actions the agent must take. The planner
can backtrack and try another path during search if a goal cannot be
reached while being capable of discarding unachievable goals.

Traditionally robot execution has relied on dispatching commands at
precise times. Such linear sequences of precisely timed commands give
no ability to adjust execution on the basis of sensory
information. Although some commands can issue tests on sensor
readings, these tests have the objective of verifying whether expected
execution conditions are occuring. If not, the state of the system is
declared off-nominal and execution of the sequence of commands is
interrupted. More recently, executives have been proposed and
implemented that significantly broaden the way robots can be commanded
\cite{mus98,alami:1998p820}. For example, the Remote Agent executive
interpreted a \textit{temporally flexible plan} which represents each
start time as a variable and contains an explicit network of bounded
delay constraints between such variables.

\begin{figure}[!t]
\centering
\includegraphics[scale=0.35]{figs/flexible-timelines.pdf}
\caption{\small Tokens with flexible temporal intervals and parametric
  constraints between tokens. This example shows the triggering of a
  water sampler based on a feature threshold while the vehicle
  Yo-Yo's. The \texttt{Waypoint\_Yo-Yo} token has a flexible duration,
  start \& end times.}
\label{fig:flex-timelines}
\vskip+0.1cm
\end{figure}

Unlike a traditional fixed time-tagged command sequences, such
flexible plans leave room for adaptation at execution time. When the
executive considers when to start a task, it propagates information
through the constraint network, computes a time bound for the
variable, selects an actual execution time within the bound, and
starts the task at that time. Temporally flexible plans therefore,
express a \textit{range of possible outcomes} of the robots
interaction with the environment within which the executive can elect
at run time the most appropriate one for the actual execution
conditions. The fact that constraints are explicitly represented
ensures that through constraint propagation the executive will respect
global limits expressed in the plan (e.g., don't start a task until a
certain condition has been satisfied but still satisfying some global
deadline). Such flexibility is critical when dealing with dynamic
ocean conditions where precise timing of a robotic action might be
indeterminate. Further, the advantage of flexibility can be contrasted
with the consequences of the intrinsic inflexibiliy of traditional
command sequences. Because they are inflexible, sequences must
necessarily be designed considering worst case scenarios.

\eu uses a \emph{state variable} representation to describe the
evolution of state over time. The instantiated history of such state
variable evolution over a temporal horizon we call \emph{timelines}
and which represent a single thread in the execution of a concurrent
system. At any given time each thread can execute a single procedure.
Thus each timeline consists of a sequence of procedures which
encapsulate and describe state evolution; we call these instantiated
atomic entities \emph{tokens}.  A token therefore describes a
procedure invocation, the state variables on which it can occur, the
parameter values of the procedure, and the time values defining the
interval. We allow encapsulation of uncertainty within these tokens
with a range of start and end times and parameters, all of which are
encoded as variables. A constraint solver in turn manipulates these
variables defined in a \eu domain model. For example, consider
a scientific need to take a water samples 100 meters from a hotspot
while an AUV is performing a Yo-Yo in the water-column. Two samples
are needed if the feature's signal is above a threshold; one
otherwise. The token that is capturing the sensory threshold has a
parametric constraints to the token which fires the requisite water
sampler. In addition, the start time of the water sampling procedure
is highly dependant on the variability of sub-sea currents and actual
speed of the vehicle. Therefore a number of values are possible for
the start times and duration of the sampling all of which are valid
combinations for desired outcomes.


\begin{figure}
\centering
\includegraphics[scale=0.3]{figs/Allen-algebra.pdf}
\caption{\small Temporal relations defined within the planner are
  based on \texttt{Allen Algebra} relations shown above.}
\label{fig:allen-algebra}
\vskip-0.3cm
\end{figure}

\subsubsection{\eu Plan Representation}
\label{sec:europa:pr}

This section describes the fundamental entities within \eu which are
typical to other CAIP-based formalisms.

\begin{description}

\item[\textbf{Variables}] Values that need to be represented to
  describe the problem domain and over which we may want to specify
  constraints. In the Shopping Agent example, the times at which the
  agent needs to leave or be back, or executes a purchase, are all
  instances where variables representing time would be used. 
  As explained in the CP section, a variable can take values from a domain. In \eu, domains are defined over a specific data type, the primitive data types supported are:
  \begin{enumerate}
    \item \textit{String}: sequences of characters, like "Red", "Spirit", etc 
    \item \textit{Boolean}: {True,False}
    \item \textit{Numeric}: integers and floats
    \item \textit{Object}: reference to an object instance (see below for a description of objects in \eu)
  \end {enumerate}
  Two types of variable Domains can be defined over a data type: 
  \begin{enumerate}
    \item \textit{Enumerated}: Finite sets over any data type, specified explicitly, for instance: ["Red","Yellow","Green"], [1,18,32], etc
    \item \textit{Interval}: Only for Numeric data types, they are specified by a [LowerBound,UpperBound] pair, for instance: [1,10], [5.0,100.0], etc.
  \end {enumerate}
  \eu's ability to represent and reason over numeric intervals is a very useful characteristic that allow users to create flexible plans. \comment{say more about the benefits of flexibility?}

\item[\textbf{Constraints}] valid plans in most problem domains have to satisfy a number of restrictions, for instance, a Shopping Agent may only be able to perform one task at a time, may have limited time or an energy or fuel budget to perform a task, and so on. In \eu, these restrictions are represented by Constraints, which can be defined over any combination of Variables in a domain. \eu provides a Constraint Library that already implements many useful constraints (temporal, resource, relational, etc), it is also possible to add new constraints if required by a particular domain.

\item[\textbf{Objects}] The items we wish to describe and refer to in
  a domain are considered Objects. As in the case with object-oriented
  analysis and design, one can seek out the nouns in any domain
  description to find likely objects. In the Shopping Agent example,
  we might consider the Agent itself, Products and Locations all to be
  objects. Objects have state and behavior. For example, a Shopping
  Agent can have:

  \begin{itemize}

  \item State: its location, a bag that contains the products it has
    already purchased (the bag can in turn be another object), etc.

  \item Behavior: go to a location to look for a product, perform a
    purchase, return home, etc.

  \end{itemize}

  Objects that have similar state and behavior can be
  described generically in terms of object types or classes. \eu
  allows the definition of classes in the same way that it is
  done in popular object-oriented programming languages, where a class 
  can have attributes (represented by Variables), and classes can be arranged in a single-parent hierarchy through inheritance. 
  However, one key aspect needs to be introduced to class definition in \eu that is not found 
  in object-oriented programming:  to describe state and behavior for the purposes of planning we need to
  build on the formalism of first order logic as explained below.

\item[\textbf{Tokens}] In first order logic, a predicate defines a
  relation between objects and properties \comment{citation?}.  In
  \eu, we define such relations between variables whose domains are
  sets of objects and sets of properties to describe state and
  behavior. For example, we might use a predicate \texttt{At($a$,$l$)}
  to indicate that agent $a$ is at location $l$, or a predicate
  \texttt{Buy($a$,$p$}) to indicate that agent $a$ is taking acting to
  buy product $p$. Note that $a$ is a variable which may have a number
  of possible values in a problem with multiple agents. Similarly, $l$
  and $p$ are variables whose values are the set of possible positions
  and products respectively. \eu can be used to create partial plans,
  where the domains of variables $a$ and $l$ can have more than one
  possible value in them, or grounded plans, where single values will
  be specified for each variable as we saw in the case of the
  N-queen's problem.

  In general, to come up with an executable plan it is not sufficient
  to state predicates that describe required state or behavior without
  also specifying some temporal extent over which each of those
  predicates hold. Predicates that are always true can be thought to
  hold from the beginning to the end of time. However, in practice,
  the temporal extent of interest must be defined with timepoints to
  represent its start and end. So we might want to write
  \texttt{At($a$,$s$,$e$,$l$)} to indicate that the agent $a$ is at
  location $p$ from time $s$ to time $e$. In fact, this pattern of
  using such predicates to describe both state and behavior of objects
  is so prevalent in \eu that we have introduced a special construct
  called a Token which has the built in variables to indicate the
  object to which the statement principally applies and the timepoints
  over which it holds. A Token is an instance of a predicate that
  represents and object's state or behavior and is defined over a
  temporal extent. In \eu, all predicate instances are Tokens. Also,
  in the same way that objects are described generically by classes,
  in \eu Tokens can be described generically by Token Types. Every
  token has five built-in variables:

  \begin{enumerate}

  \item \textit {start}: The beginning of the temporal extent over
    which the predicate is defined.

  \item \textit {end}: The end of the temporal extent over which the
    predicate is defined.

  \item \textit {duration}: The duration of the temporal extent over which the predicate is defined. The constraint \textit{start} $+$
    \textit{duration} = \textit{end} is enforced automatically.

  \item \textit{object}: The set of objects to which a token might
    apply. In a grounded plan each Token applies to a specific Object,
    reflecting the intuition that we are using Tokens to describe some
    aspect of an Object (i.e. its state or behavior) in time. However,
    in a partial plan, the commitment to a specific object may not yet
    have been made.

  \item \textit{state}: Tokens can be \texttt{ACTIVE, INACTIVE,
      MERGED}, or \texttt{REJECTED}. The state variable captures the
    token's current state and its reachable states through further
    restriction.  This is required to support CAIP's approach to planning where goals and subgoals can be satisfied by either adding new intervals (Tokens) to the plan, or by merging with existing intervals. Token State lifecycle is examined in detailed in a separate section below.

  \end{enumerate}

\item[\textbf{Built-in Object Types}] In a domain model, there may be some classes that don't have any time-dependent state or behavior, and therefore those classes will not have any Token types associated with them, for instance in the Shopping Agent model, the set locations and products are static for the agent's purposes so we may only want to say what they are but not have any state or behavior associated with them.

However, the most common case in any non-trivial domain model is that Objects will have many associated Tokens in order to describe their state and behavior throughout a plan. There are a couple of Object Types that are so commonly used in domain descriptions, that \eu provides a built-in implementation for them:

\begin{enumerate}
    \item \textit{Timelines}: Often objects in a domain must be described by exactly one
    token for every given timepoint in the plan. Any instances of a class derived from \eu's built-in
    Timeline class will induce ordering requirements among its tokens
     to ensure no temporal overlap may occur among them  \cite{mus94}. 
     \comment{add small example?}

    \item \textit{Resources}: Metric resources, e.g. the energy of a battery or the
    capacity of a cargo hold, are objects with an explicit quantitative
    state in time and with a circumscribed range of changes that can occur
     to impact that state i.e. produce, consume, use, change. 
     Resources are such a common requirement for \eu users that built-in object types (with their corresponding token types to denote production, consumption, etc) are provided for them. Instances of classes derived from a Resource will induce
     ordering requirements on their Tokens in order to ensure that the
     level of the resource remains within specified limits.
     \comment{add small example?}
\end{enumerate}

\item[\textbf{Token State Model}] In \eu's representation, goals are temporally scoped states that need to be achieved, or actions that must be performed. As a result, \eu internally represents goals as Tokens.  Goals can be posted explicitly, for instance, when a user asks the Shopping Agent to own a specific product by a specific time, or implicitly when subgoals are created as a result of domain compatibilities (described in the CAIP framework above), for instance, when the Shopping Agent must get to a Location where a Product is available before it is able to purchase it, in this case the original goal of purchasing the product will spawn a subgoal of being at a specific location.

Looking at a how a simple Shopping Agent goal would be addresed by \eu helps illustrate how Token States support the planning process.

Let's assume we start with a partial plan that places the Shopping agent at home at time 0 and with no Products in its possession, then we post a goal for the Shopping Agent to own a Drill at time 10.

\comment{Add figure showing an AgentLocation timeline with an At(Home) token with start=0 and end open, an empty AgentBag timeline, and a Own(Drill) token with start=10 floating below both timelines}

Notice that the \texttt{Own(Drill)} token that represents the goal is not immediately placed on the \texttt{AgentBag} timeline since \eu must determine whether it can achieve that goal and how. When the goal is posted and before any decisions are made on it, the  \texttt{Own(Drill)} token is created in an INACTIVE state, then \eu looks at the current partial plan and considers 2 alternatives:

\begin{enumerate}
    \item If there are any tokens in the plan that are compatible with the new goal, \eu may attempt to satisfy the goal by merging it with any of them (if there is more than one it can try them one at a time), in this case, the state of the token would change to MERGED.
    \item Instead of merging with existing tokens in the plan, \eu may decide to insert the new token into the plan, in that case its state would change to ACTIVE.
\end{enumerate}

When a token is merged, all restrictions imposed
on the merged token are transferred to the active token upon which it is merged. Merging a
token requires finding a target active token that is compatible with
the inactive token. Two tokens are merge-compatible if they are instances of the same predicate and
that no intersections between corresponding variables are empty. The
effect of merging is illustrated in Figure 3.

\comment{Add figure showing the effect of merging}

When a token is activated all compatibilities associated with its token type are evaluated, which may lead to new subgoals being generated. In our example, there should be a compatibility stating that the Shopping Agent must be at the same location as the product it is purchasing. In that case, activating the \texttt{Own(Drill)} token will cause a new  \texttt{At(Drill.location)} to be created in an INACTIVE state, which in turn will have to be dealt with.

\comment{Add figure showing the effect of activating the token by putting it on the AgentBag timeline, and a new At(Drill.location) token floating below}

If neither activating nor merging works, a token may become REJECTED, this is possible only for goals posted by the end user and specified as optional. Subgoals generated as a result of token activation cannot be rejected, if a subgoal cannot be satisfied through activation or merging then it will cause the original activation decision that spawned it to be retracted (\eu's search mechanisms are explained in detail in a separate section later on).

As we can see, a token's state can go from INACTIVE to ACTIVE, MERGED or REJECTED, this lifecycle is illustrated in the figure below.

\comment{Add figure showing possible state transitions}

\item[\textbf{Rules}] In order for a plan to be valid, it must comply with all rules and
regulations pertinent to the application domain in question. Rules
govern the internal and external relationships of a token. For
example, consider a parameterized action \texttt{Go(origin,destination)} which describes
a Shopping Agent traveling from one location ( \textit{origin}) to another ( \textit{destination}). The
parameters are instantiated on a token as variables whose domain of
values is the set of all locations in a given problem. A rule
governing an internal relationship among token variables might
stipulate that a transition must involve a change in location. This
can be easily expressed as a constraint on the definition of a
predicate of the form: \textit{origin != destination}. It makes sense to further
stipulate that the agent must be located at  \textit{origin} before travel can happen
and the agent must be located at  \textit{destination} when completed. This is
an example of a rule governing an external relationship among
tokens. It specifies a requirement that tokens of the predicate
\texttt{At} that represent the location state for the Shopping Agent 
precede and succeed tokens of the predicate \texttt{Go} that represents traveling.

\comment{ Add figure showing internal and external relationships from rules on action Go}

Figure xx illustrates the entities and relations involved in specifying
such a rule on a \texttt{Go} action. The token on which the rule applies
is referred to (\texttt{Go}) as the master. Each  \texttt{At} token required by the
master is referred to as a slave. All variables are indicated by name
and their domains are expressed as intervals in the case of temporal
variables and as enumerations for the remainder. Application of a rule
on a token can thus cause slave tokens, variables, and constraints to
be introduced.  

The capability of a domain rule to cause a slave token to be created
is a key vehicle through which planning occurs. Semantically, this
rule imposes a requirement for supporting (slave) tokens to be in the plan in
order for the master to be valid. 

\comment {talk about condition/effect annotations?}


\end{description}

We have just covered the main elements of \eu's planning representation:

\begin{enumerate}
   \item \textbf{Domains, Variables and Objects} to represent the entities (Objects) and attributes (Variables) needed to represent a problem domain	
    \item \textbf{Tokens} Temporally scoped predicates to represent states and actions in time. 
    \item \textbf{Token States} which are necessary for the implementation of planning operations on a partial plan.
    \item \textbf{Constraints} to describe relationships among attributes of objects and tokens. \eu has a built-in library of constraints that covers time, resources and others that are common in real life problem domains.
    \item \textbf{Domain Rules} to describe internal and external relationships on and
     between tokens respectively. Rules are applied to active tokens (master tokens) 
     referred to as masters and typically produce inactive tokens (slaves) which are a key vehicle through which the planning
     process evolves. \comment{citation, or perhaps remove this statement?:}The rule structure in \eu differs from the more restrictive commitment to preconditions and effects in classical
     planning which prohibits durative actions and disembodied effects (i.e. effects which may not occur until some temporal distance after
     the end of the action).    
    \item \textbf{Timelines and Resources} are built in implementations of Object types that appear frequently in real life domains.
\end{enumerate}



%\begin{figure*}
%\centering 
%\subfloat[]{\label{fig:plan-evolve1}\includegraphics[width=0.3\textwidth]{figs/Plan-evolve-1.pdf}} 
%\subfloat[]{\label{fig:plan-evolve2}\includegraphics[width=0.3\textwidth]{figs/Plan-evolve-2.pdf}} 
%\subfloat[]{\label{fig:plan-evolve3}\includegraphics[width=0.3\textwidth]{figs/Plan-evolve-3.pdf}} 
%\caption{\small An illustrative plan synthesis example with concurrent
%  timelines. Each timeline is an instantiation of a subsystem tracked
%  by the planner. Tokens describe the instantiated state of subsystem
%  at a particular time instance and enforce temporal and parametric
%  constraints. Abstract goals are decomoposed into successively less
%  abstract tokens and instantiated in co-temporal timelines using
%  \texttt{Allen Algebra} relations. All tokens represent flexible
%  start/end times. \ref{fig:plan-evolve1} shows an initial state
%  evolving into \ref{fig:plan-evolve2} and \ref{fig:plan-evolve3}.}
%  \label{fig:Plan-evolve}
%  \vskip-5pt
%\end{figure*}

%Fig. \ref{fig:Plan-evolve} shows an illustrative example. We show four
%timelines which track \texttt{Goal, Path, Navigation} and
%\texttt{Command} state over time. Tokens on the goal timeline indicate
%a survey within a bounded volume and a min/max depth envelope. As the
%Volume Survey sub-goals on the \texttt{Path} timeline, the initial
%token on that timeline gets \emph{squeezed} to have dependencies of
%the goal token instantiated. Initially this dependency are the tokens
%\texttt{Go(x1,y1)} and \texttt{Go(xn,yn)} indicative of the area of
%coverage. The first of these in turn generates a sub-goal on the same
%timeline to go to the next intermediate waypoint, which is illustrated
%in Fig. \ref{fig:plan-evolve2}. Subsequent sub-goals (not shown in the
%figure) ultimately generate all the tokens to completion for this
%timeline. Fig. \ref {fig:plan-evolve3} shows a snapshot further along
%in plan generation and shows additional subgoals on the \texttt{Path,
%  Navigation} and \texttt{Command} timelines. Each token can thus
%trace its causality when instantiated in the plan. The domain model
%provides source of these temporal and parametric dependencies in the
%form of rules. 

%\begin{minipage}[c]{\textwidth}
%\vspace{+0.5cm}
% \framebox[\textwidth][t]{
%Volume\_Survey(x$_1$,y$_1$,x$_n$,y$_n$,Min\_Depth, Max\_Depth)\\
%$\Rightarrow$ \\
%met\_by Go(x$_1$,y$_1$,Min\_Depth); \\
%starts [50,0] Go(x$_2$,y$_2$,Min\_Depth,Max\_Depth); \\
%\ldots{} \\
%ends [50,0] Go(x$_{n-1}$,y$_{n-1}$, Min\_Depth,Max\_Depth); \\
%meets Go(x$_n$,y$_n$,Min\_Depth, Max\_Depth); 
%\end{minipage}

% \end{enumerate}



% \subsubsection{Partitioned Control}


% \subsubsection{State Estimation}

% Even as automated reasoning approaches have the ability to dynamically
% retarget the vehicle, estimating environmental signals of interest (or
% their proxies) is important to be able to enable opportunistic science
% in the water-column. In \texttt{T-REX} feature recognition revolves
% around a Hidden Markov Model (HMM) \cite{rabiner86} which is encoded
% directly within the unified representational and computational
% framework of a reactor. HMMs are useful since the stochastic nature of
% these models can correlate the type of features we want to detect with
% the sensor observations. Online targeted sensor data is classified by
% apriori generated cluster data which in turn determines the
% probability of having seen the feature of interest. Together with
% posterior probability, the HMM generates a probability of being within
% the target feature if it dominates the distribution. If other sampling
% conditions are satisfied (for instance sampling proximity), a
% constraint is activited in the model which can trigger a water sampler
% and also preserve the memory of sampling to alter a future transect
% resolution.


% The HMM is customized for the feature of interest and is built offline
% using clustering techniques from large data sets; we use Kohonen's
% Self Organizing Maps \cite{kohonen}. We use a semi-supervised learning
% method \cite{zhu05} to extract an environmental model that make uses
% of both labeled and unlabeled data for training. Online,

% The sensor data for classification depends on the
% feature of interest; for detecting INLs we use backscattering data
% from a HydroScat and for tracking riverine plumes to track fertilizer
% runoff, we use an ISUS Nitrate sensor.

%%% Local Variables: 
%%% mode: latex
%%% TeX-master: "ieee-ram09"
%%% End: 


\section{Experimental Results}
\label{sec:results}

% Show some of the results of the use of T-REX to convince people that is for
% real. CANON will be the predominant driver of the work.

\section{Future Work}
\label{sec:future}

% Where do we see all this going esp. in the context of embedded AI Planning
% for marine robots? This can be (and should be)
% speculative. Shore-side/onboard autonomy coordination and control of
% multiple vehicles is an important result of this activity.

% 1.6 Future Work?
% explain where EUROPA is going? (ANML, incorporation of classical planning algorithms, soft constraints, better search, etc).

% The main advantage of CP is the richness of the constraints that can be modeled and the explicit use of logical inference to prune the search space



\input{conclde.tex}

\section{Acknowledgements}

The authors are funded by a block grant from the David and Lucile
Packard Foundation to MBARI. 

\bibliography{references}

\end{document}
