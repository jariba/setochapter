\section{Fundamentals of  Automated Planning}
\label{sec:europa}

{\footnotesize
  \begin{quote}
[Planning] \emph{is an abstract, explicit deliberation process that chooses and
organizes actions by anticipating their expected outcomes. This
deliberation aims at achieving as best as possible some prestated
objectives. Automated planning is an area of Artificial Intelligence
(AI) that studies this deliberation process computationally.} -- from
\textbf{Automated Planning Theory and Practice} by Ghallab, Nau and
Traverso \cite{ghallab04} 
\end{quote}
}

% Planning, for our purposes, can be thought of as determining all the
% small tasks that must be carried out in order to accomplish a goal.
To articulate the fundamentals of automated planning briefly and use
that to motivate the mechanisms we use in our specific form of the
technique we start with a simple example.

\begin{quotation}

  In the near future, a personal robot sets out to buy a gallon of milk
  This involves a number of tasks: obtain keys, obtain wallet,
  start car, drive to store, find and obtain milk, purchase milk, etc.
  The embedded planner has to have a ``model'' of the world in which it
  lives and has to use the task primitives in this model to structure
  the actions so it achieves its goal. Constraints control when certain
  tasks can or cannot occur. For example the robot must obtain the keys
  and wallet \emph{before} driving to the store and pick up the milk
  \emph{before} purchasing it.

\end{quotation}

For such a robot the milk buying plan at the store might look like
\comment{Figure needed}.


% \eu is a general purpose AI planning toolkit developed at NASA Ames.  
% TODO: Talk about NASA missions where \eu has been used.

\eu is a versatile planning framework which can be used in a range of
problem-solving methods. 

\paragraph \texttt{Constraint Satisfaction}: A canonical problem in
dealing with constraints is the $N$-Queens problem in which chess queens
must be placed on an  $N$x$N$ chessboard so no queens attack the other. Fig.
\ref{fig:nqueens-1} shows an example of a random positioning of queens
on a  $N$x$N$ chessboard. Queens in violation of the non-attack constraint
are highlighted in red.

\begin{figure} \centering
  \includegraphics[scale=0.35]{figs/Example-NQueens0.jpg}
  \caption{\small N-Queens problem. Queens in violation of the
    non-attack constraint are highlighted in red.}
\label{fig:nqueens-1}
\vskip+0.1cm
\end{figure}


If we define $N$x$N$ variables Q$_{rc}$, $r \in [1,N]$, $c \in [1,N]$,
Q$_{rc} = 1$ if cell $r,c$ in the chessboard is occupied by a Queen, $0$
otherwise. Then the following constraints need to be satisfied:

\begin{equation}
 Sum(Q_{rc})= \left\{
\begin{array}{l l}
  1 & \forall r \quad \mbox{(only one Queen per row)}\\
  1 & \forall c \quad \mbox{(only one Queen per column)}\\ 
\end{array} \right.
Sum(Q_{r+i,c+i}) = 1\\
Sum(Q_{r-1,c+i}) = 1 \quad \mbox{(only one Queen on each diagonal)}\\
\end {equation} 

\comment{fix newline and numbering problem above}

The problem can be solved by finding assignments for all variables
Q$_{rc}$ that satisfy the above constraints. Fig. \ref{fig:nqueens-2} is
a solution found by \eu using that formulation and a specialized search
procedure.

\begin{figure}
\centering
\includegraphics[scale=0.35]{figs/Example-NQueens1.jpg}
\caption{\small N-Queens solution generated by \eu}
\label{fig:nqueens-2}
\vskip+0.1cm
\end{figure}


\paragraph \texttt{Scheduling}: In the Resource Constrained Project
Scheduling Problem (RCPSP) \comment{citations needed}, a project
consisting of a set of activities must be scheduled in a way that
satisfies minimum and/or maximum temporal separation constraints. The
activity schedule must also respect fixed limits on the availability of
resources required to perform each activity. In addition to satisfying
temporal and resource constraints, it is common for the user to want to
minimize makespan so that the entire project is finished as early as
possible. Fig. \ref{fig:rcpsp-1} shows an example of a a solution
provided by \eu for an RCPSP instance with 10 activities, 5 resources
and 30 temporal constraints.

\begin{figure}
\centering
\includegraphics[scale=0.35]{figs/Example-UBO0.jpg}
\caption{\small A \eu solution to an RCPSP \comment{citation} problem.}
\label{fig:rcpsp-1}
\vskip+0.1cm
\end{figure}


\paragraph \texttt{Planning}: In the Shopping Agent Problem
\cite{russelnorvig} an agent needs to purchase a set of products (milk,
drill, etc) that are available at specific locations (supermarket,
hardware store, etc), the agent is subject to temporal (must complete
tasks by specific deadlines) and resource (fuel, carrying capacity, etc)
constraints. The agent needs to figure out what actions need to be
performed to find and acquire the required items, as well as when to
perform each of those actions. Fig. \ref{fig:shopping-1} shows a
solution produced by \eu for such a problem instance.

\begin{figure}
\centering
\includegraphics[scale=0.35]{figs/Example-Shopping0.jpg}
\caption{\small A \eu solution to a shopping agent problem domain where
  the agent needs to buy Bananas, Milk and a Drill.}
\label{fig:shopping-1}
\vskip+0.1cm
\end{figure}

As the examples above show, a planning problem (actions to achieve a
goal) may embed a scheduling problem (what resources are necessary to
achieve stated goals) and both planning and scheduling could embed a
Constraint Satisfaction problem. % since temporal, resource and  other
% kinds of constraints must be respected in most real life problems.
The relationships between Planning, Scheduling and Constraint
Satisfaction have been examined \cite{smith00} and have lead to use of
constraint reasoning in \eu as its innermost building block. % Below we
% examine the main ideas from Constraint Satisfaction Programming, and
% from Constraint Based Planning and Scheduling that constitute
% \texttt{EUROPA}$_2$'s theoretical underpinnings.

\subsection{Constraint Programming}
\label{sec:europa:cp}

Constraint Satisfaction Programming (also known as Constraint
Programming (CP)) is a discipline that provides a generic framework for
representing, solving and making logical inference on constraints. A
complete treatment of this discipline is given in [MAR98] and [APT03]
and very concise introductions are provided in [BAR99] and
[LUS01].\comment{fix citations}

A constraint programming problem consists of a set of variables $V=
{x_1,..,x_N}$, where each variable takes values from a domain
$d_1,..,d_N$; in this chapter we will deal only with discrete and
finite domains. Given a defined conjunctive set of constraints on the
variables: $C=\{c_1(x_1,..,x_N), ..., c_K(x_1,..,x_N)\}$, the objective
is to find one or more value assignments to $V$ where all constraints
are satisfied.

To solve a problem, CP techniques use logical inference to perform two
important operations:

\begin{enumerate}

\item \textbf{Bounds propagation}: To infer upper and lower variable
  bounds. For example, from the constraints $x_1 + x_2 \leq 2$ \&
  $x_i \geq 0$, we can infer $[0,2]$ bounds for $x_1$ and $x_2$

\item \textbf{Domain reduction}: To infer a valid set of values for a variable.
  For example, for  constraints allDifferent($x_1,x_2,x_3$), $x_i \geq
  0$ \& $x_i \leq 4$, if $x_1 = 1$ and $x_2 = 3$ we can infer that the
  valid domain for $x_3$ is reduced  to $\{2,4\}$

\end{enumerate}

CP is normally implemented as part of a programming language and
constraints are normally represented as objects [PUG95]
\comment{citation}. Any constraint can be introduced and, as long as the
bounds propagation and domain reduction protocols specified by the host
CP system are enforced, it will be indistinguishable from any other
``primitive'' CP constraint, such as $\leq$ or $\geq$. In theory,
solving a CSP problem is NP-Hard \cite{ghallab04}, but often very
efficient in practice using a number of algorithms and techniques that
are well understood.

\comment{TODO: talk about arc consistency algorithms, computational
  complexity and solvers.}

\subsection{Constraint-Based Attribute and Interval Planning}
\label{sec:europa:cp}

The most common planning formulations use a propositional
representation, where the state of the world is represented by a set
of propositions (statements that can be true or false), and operators
change the truth values of these propositions \cite{gen87}. Although
these formulations are powerful and have allowed researchers to
develop numerous contributions in automated planning, there are many
classes of problem domains that are difficult to represent using this
formalism. In particular, it is hard to represent time, resources,
mutual exclusion and concurrency using propositions \comment{citation
  needed}. While CSP representations have traditionally had an edge in
formulating and representing planning problems, 
% It is straightforward to represent and reason about all of
% those elements using a CSP representation, as a result there has been
% some work on doing automated planning while taking advantage of CSP
% (TODO: ref). Traditionally, the entire planning problem is translated
% into a CSP and then solved using traditional CSP methods, this
leading to a formulation where action choices and relationships are
represented as variables and/or constraints, there are at least two
major drawbacks:

\begin{enumerate} 

\item Given that action choices and relationships (rules) are expressed
  through variables, the domain descriptions that result from this
  approach are not intuitive and therefore difficult to understand and
  debug

\item If the structure of a planning model (actions, conditions,
  effects, dependencies at the action level) is not explicitly
  maintained by the CSP planner, the search algorithms are deprived of
  critical information to make better decisions. If that structure is
  maintained (for instance, by internally marking variables that
  represent action choice and relationships between actions), it is
  still hard to write search algorithms as any planning-specific
  insight has to be translated into the CSP representation of
  variables and constraints \comment{For example?}

\end{enumerate}

Constraint-based Planning and specifically Constraint-Based Attribute
and Interval planning (CAIP) \cite{mus94,frank2003} is intended to
close that gap. On one hand, it takes advantage of a CSP
representation; one the other it also uses attributes and intervals to
maintain an explicit representation of the elements relevant to
planning. This makes it easier to write algorithms that search for an
reason abut plans. The main elements of CAIP are:

\begin{enumerate}
	\item Intervals: \comment{needs to be defined}
	\item Attributes:
	\item Domain Constraints an Configuration rules:
\end{enumerate}

\comment{Explain how planning problem instances and their solutions are represented}

\subsection{EUROPA's Plan Representation}
\label{sec:europa:pr}


EUROPA is an implementation of the CAIP framework explained in the previous section



Variables

Values that need to be represented to describe the problem domain and over which we may want to specify constraints.

Objects

The things we wish to describe and refer to in a domain are considered Objects. As in the case with object-oriented analysis and design, one can seek out the nouns in any domain description to find likely objects. In a Mars rover example, we might consider the satellite, the camera, and the attitude controller (which orients the robot for taking pictures) to be objects. Objects have state and behavior. For example, a camera can be:

    off,
    ready, or
    taking a picture. 

An attitude controller can be:

    pointing at a position, or
    turning from one position to another. 

An object is an instance of a class. In EUROPA we model using the abstraction of a class to speak about all instances having certain properties of state and behavior. In order to describe such state and behavior we build on the formalism of first order logic.


Temporally Qualified Predicates and Actions

A predicate defines a relation between objects and properties. In EUROPA, we define such relations between variables whose domains are sets of objects and sets of properties. For example, we might use a predicate Pointing(a,p) to indicate that a rover's attitude controller a is pointing at position p. Note that a is a variable which may have a number of possible values in a problem with multiple satellites. Similarly, p is a variable whose values are the set of possible positions. In a grounded plan, single values will be specified for each variable as we saw in the case of a solved Constraint Satisfaction Problem.

In general, is is not sufficient to state that a predicate is true without giving it some temporal extent over which it holds. Predicates that are always true can be thought to hold from the beginning to the end of time. However, in practice, the temporal extent of interest must be defined with timepoints to represent its start and end. So we might want to write Pointing(a,s,e,p) to indicate that the attitude controller a is pointing at position p from time s to time e. In fact, this pattern of using such predicates to describe both state and behavior of objects is so prevalent in EUROPA that we have introduced a special construct called a Token which has the built in variables to indicate the object to which the statement principally applies and the timepoints over which it holds. In EUROPA, all predicate instances are Tokens.


Tokens

A Token is an instance of a predicate or an action and is defined over a temporal extent. Every token has five built-in variables:

    start: The beginning of the temporal extent over which the predicate is defined.
    end: The end of the temporal extent over which the predicate is defined.
    duration: The constraints start + duration = end is enforced automatically.
    object: made. The set of objects to which a token might apply. In a grounded plan each Token applies to a specific Object, reflecting the intuition that we are using Tokens to describe some aspect of an Object (i.e. its state or behavior) in time. However, in a partial plan, the commitment to a specific object may not yet have been made.
    state: Tokens can be ACTIVE, INACTIVE, MERGED, or REJECTED. The state variable captures the token's current state and its reachable states through further restriction. See the Token State Model discussion below for details. 

Objects (continued)

As we have noted, Tokens describe some aspect of an Object in time. Objects thus may have many Tokens in a plan in order to describe their state and behavior throughout all points in time of the plan. Within this general framework, we note a few particulars:

    Static Facts: Classes without predicates lead to Objects without Tokens. This arises where an objects state or behavior is independent of time. For example, a domain may have a set of locations and/or paths for which there is nothing more to say than that they exist.
    Timelines: Often objects in a domain must be described by exactly one token for every given timepoint in the plan. Such objects are so common that we provide a special construct in EUROPA to extend these semantics to derived classes. Any instances of a class derived from a Timeline will induce ordering requirements among its tokens in order to ensure no temporal overlap may occur among them. See below for further discussion.
    Resources: Metric resources, e.g. the energy of a battery or the capacity of a cargo hold, are objects with an explicit quantitative state in time and with a circumscribed range of changes that can occur to impact that state i.e. produce, consume, use, change. These changes are captured as tokens. Resources are such a common requirement for EUROPA users that special constructs are also provided for them. Instances of classes derived from a Resource will induce ordering requirements on their Tokens in order to ensure that the level of the resource remains within specified limits. 

Timelines

It may be sufficient to maintain only a partial-order among tokens. However, it is often the case that tokens represent states and actions of a single object in the system. Such tokens are typically mutually-exclusive. EUROPA uses a Timeline structure developed in (??) to concisely capture system components whose behavior is described over time in this manner.

For example, consider the tire-world domain. The tire was located in the trunk with the predicate tireLocated(Trunk) and located on the ground with the predicate tireLocated(Ground). Clearly, the same tire cannot be in both places at once. A Timeline provides a simple method for aggregating the statements about a tire such that they are mutually exclusive.

Figure 1: A Timeline for a Tire

Figure 1 illustrates how a Timeline can be used to describe the whereabouts of a tire. The tire is an object in the tire-world represented as a timeline. It has predicates associated with it which can describe its states and actions over time. The predicate names need not contain the tire prefix; this is now implicit since the tokens are assigned to a specific tire instance (i.e. an instance of a Timeline). While Timelines are a useful element of the EUROPA planning paradigm, they are not essential. A more general notion of a system Object can be used which does not impose restrictions of mutual-exclusion or non-zero duration. This can be important in supporting partial-order planning.

For this example, the state of the tire is specified using 3 contiguous tokens. The end and start time-points are related by an equality constraint. A Moving predicate has been introduced to cover the transition from one location to another. It takes 2 location arguments. Notice that precedence relationships exist between tokens such that they cannot overlap but the start and end times may remain flexible. To illustrate this, sample times are included. Assume a total time range of interest between 0 and 1000. In addition, tokens on Timelines have a minimum duration of 1. As a result the values shown are the most flexible possible values for the start and end of each token. There are a number of advantages of allowing this flexibility. First, the basic structure of the plan can be developed without over-committing to specific times. If it is not necessary for the tire to be on the ground at time-step 4, then a planner should not be forced to specify it. Such an approach permits a least-commitment approach to planning. Second, in many domains it is simply impossible to know in planning exactly how long an activity might take. For example, a common activity of driving a car from one point to another on a road can take varying amounts of time depending on traffic and traffic lights. In such circumstances it is more practical to express upper and lower bounds on durations which naturally lead to intervals for start and end times. In such domains, flexibility in planning aids robustness in execution.
Rules

In order for a plan to be valid, it must comply with all rules and regulations pertinent to the application domain in question. Rules govern the internal and external relationships of a token. For example, consider a parameterized predicate describing a transition from one location to another. Let the parameters be from and to. The parameters are instantiated on a token as variables whose domain of values is the set of all locations in a given problem. A rule governing an internal relationship among token variables might stipulate that a transition must involve a change in location. This can be easily expressed as a constraint on the definition of a predicate of the form: from != to. It is reasonable to further stipulate that one must be Located somewhere before a transition can occur, and one must end up Located somewhere when completed. This is an example of a rule governing an external relationship among tokens. It specifies a requirement that tokens of the predicate Located precede and succeed tokens of the predicate Moving.

Figure 2: Internal and external relationships from rules on Moving Figure 2 illustrates the entities and relations involved in specifying such a rule on a Moving predicate. The token on which the rule applies is referred to as the master. Each Located token required by the master is referred to as a slave. All variables are indicated by name and their domains are expressed as intervals in the case of temporal variables and as enumerations for the remainder. Application of a rule on a token can thus cause slave tokens, variables, and constraints to be introduced.
Token State Model

The capability of a domain rule to cause a slave token to be created is a key vehicle through which planning occurs. Semantically, this rule imposes a requirement for supporting tokens to be in the plan in order for the master to be valid. There are 2 possibilities to consider:

    The slave is inserted as an active token in the plan. As such, rules may be activated on the slave, and it may consume resources.
    The token is merged with a matching token already in the plan. Once a slave is merged, the requirement it represents are considered satisfied. The process of merging passes on all restrictions imposed on the slave to the active token upon which it is merged. Merging a token requires finding a target active token that is compatible with the inactive token. For an active token and an inactive token to be compatible requires that they are instances of the same predicate and that no intersections between corresponding variables are empty. The effect of merging is illustrated in Figure 3. 

Figure 3: Merging an inactive token on an active token. Domain restrictions occur in highlighted variables of active token.

On creation of a token, where the commitment has not been made yet to activate or merge the token, the token is said to be inactive. There is a nuance to the state model for tokens which relate to the mode of its creation. If the token is allocated explicitly by an external actor, rather than internally through rule firing, it may include the state rejected indicating the planner is permitted to reject the token. A valid plan can include rejected tokens. This typically arises where the token represents a goal that is preferable to achieve but not mandatory. This state is not reachable for slaves since that would imply selective adherence to the domain model. Figure 4 presents the state transition diagram for token states relevant for planning. The transitions are operations on a partial plan which can decide an outcome for an inactive token. As operations which may arise in search, they must be reversible during backtracking. The cancellation operations in each case are also shown.

Figure 4: Token States and Transitions for Planning

The state of a token is embodied with a 5th and final built-in variable referred to as the state variable. The planner states are values in the domain of this variable {MERGED, ACTIVE, REJECTED}. The INACTIVE state is captured by the variable being unbound. If a value is removed from the domain, that state will not be reachable. For example, when a slave is created via a rule, the REJECTED value is removed from its state variable.
Summary

This page has presented the main elements of the EUROPA planning approach. They are:

    Temporally scoped predicates and actions (a.k.a. Tokens) to represent states and actions in time. Note that tokens do not discriminate between state and action.
    Constraints to describe relationships among tokens. This provides an expressive method of describing interactions among tokens in the context of Temporal Planning.
    Timelines as a concise abstraction to express the evolution of state and behavior for a system component. It provides semantics of mutual exclusion sequencing in time. Other core abstractions are available in EUROPA for handling metric resources.
    Domain rules to describe internal and external relationships on and between tokens respectively. Rules are applied to active tokens (master tokens) referred to as masters and typically produce inactive tokens (slaves) which are a key vehicle through which the planning process evolves. The rule structure in EUROPA differs from the more restrictive commitment to preconditions and effects in classical planning which prohibits durative actions and disembodied effects (i.e. effects which may not occur until some temporal distance after the end of the action).
    Token States which are the basis of planning operations on a partial plan. 
