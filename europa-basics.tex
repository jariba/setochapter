\section{Foundational Concepts}
\label{sec:concepts}

\rx is an adaptive, artificial intelligence based controller and
provides \kcomment{a} general framework for building reasoning systems
for real-world autonomous vehicles. At MBARI \rx is used for AUV
control; another instantiation of the system is being used for control
of a terrestrial personal robot \cite{pr2, Meeussen:2010dn}. The
development of \rx has been targeted at surveying a number of
oceanographic features which are dynamic and spatio-temporally
unpredictable. Typically this requires our AUVs to balance the goals
of spatial coverage while opportunistically following features of
scientific interest and to do so while being aware of its own resource
limitations (\eg the battery state of charge) and overall proximity to
other observational assets for obtaining scientific ground-truth. We
are interested in representational frameworks that allow robots to
pursue long-term objectives while choosing short-term gain and can
gracefully deal with exogenous or endogenous change.

To enable this responsiveness to external observations, the agent has
to be able to synthesize plans insitu and to re-plan without human
intervention. \rx uses a temporal constraint-based planner with a
demonstrated legacy of having flown on NASA space missions. Our
autonomy architecture brings three key innovations for AUV control:
in-situ plan synthesis, the use of flexible plan representations with
a sound theoretical basis and systematic compositional control with
the use of partitioned networks and doing so with a demonstrated
capability at sea \kcomment{for} novel ocean observation methods.

The next few sections will cover \eu in a bottom-up fashion, going
from its \kcomment{general} theoretical framework, to the specific
representation, modeling and algorithms implemented by it.  A brief
introduction to Constraint Programming and Constraint-Based Planning
is provided followed by the specific representation that \eu uses to
undertake Constraint-Based Planning.  Subsequently, we provide an
overview of \eus architecture and how it can be embedded in a planning
application such as \rxe. Finally, \eus modeling, inference and search
mechanisms are covered in detail so that the reader can
\kcomment{appreciate} how \rx can perform \kcomment{deliberation while
  being responsive to events in the real-world}.

\subsection{Automated Planning}
\label{sec:planningfound}

{\scriptsize
  \begin{quote} [Planning] \emph{is an abstract, explicit deliberation
      process that chooses and organizes actions by anticipating their
      expected outcomes. This deliberation aims at achieving as best
      as possible some prestated objectives. Automated planning is an
      area of Artificial Intelligence (AI) that studies this
      deliberation process computationally.} -- from \textbf{Automated
      Planning Theory and Practice} by Ghallab, Nau and Traverso
    \cite{ghallab04}
\end{quote}
}

Planning as an activity is ingrained in human decision-making
\cite{berthoz} and its generalization into a computational framework
demonstrating machine intelligence \kcomment{has for long been}
targeted by pioneers in computer science
\cite{computersthought}. Early work in this context focused on logical
theorem proving by computer \cite{green69} which has formed the basis
of much of the discipline. The emphasis in these early efforts was to
logically derive the existence proof of a trajectory of actions for a
hypothetical robot (moving blocks, chess pieces, furniture etc) to
achieve a specific goal given specific initial conditions. Two
important properties of \emph{Soundness}\footnote{A planning algorithm
  is sound if invoked on a problem $P$ returns a plan which is a
  solution for $P$.} and \emph{Completeness}\footnote{A planning
  algorithm is complete if invoked on a solvable problem $P$ is
  guaranteed to return a solution.} along with derived results for
algorithmic complexity \cite{gareyjohnson,corman} were a defining
cornerstone for understanding how different computational paradigms
could actually be tractable first in simulation and subsequently in
actual robotic platforms. In this chapter our emphasis is to focus on
embedded AI Planning and the execution of plans on an embedded
\kcomment{marine robotic} platform. The techniques we highlight are
general purpose as noted in Section \ref{sec:intro}.

\begin{figure}[b]
\centering
  \includegraphics[scale=3]{figs/shopping_europa.jpeg}
  \caption{\small A potential plan for a robot shopping for groceries.}
\label{fig:shop:shopping}
\end{figure}

To articulate the fundamentals of automated planning briefly and use
that to motivate the mechanisms we use in our specific form of
planning we start with an example.

{\small
  \begin{quote}
    In the near future, a personal robot sets out to buy a gallon of
    milk. This involves a number of tasks: obtain keys, obtain wallet,
    start car, drive to store, find and obtain milk, purchase milk,
    etc.  The embedded planner has to have a ``model'' of the world in
    which it 'lives' and has to use the task primitives in this model
    to structure the actions so it achieves its goal. Constraints
    control when certain tasks can or cannot occur. For example the
    robot must obtain the keys and wallet \emph{before} driving to the
    store and pick up the milk \emph{before} purchasing it.
\end{quote}

For such a robot the milk buying plan at the store might look like
that shown in Fig. \ref{fig:shop:shopping}.

At the core of the \rx framework, is the deliberation engine, \eue,
which has a rich legacy from NASA missions \cite{mus98,rajan00,
  jonsson00,aichang04, bresina05}. \eu is a versatile Constraint-based
Temporal Planner which continues to be deployed on a diverse set of
applications including recently, planning for the International Space
Station \cite{barreiro09}. We first motivate this section with some
problem domains this planner has handled.

\paragraph{} {\em Constraint Satisfaction}: A canonical problem in
dealing with constraints is the $N$-Queens problem in which chess
queens must be placed on an $N \times N$ chessboard so no queen can
attack the other. 

If we define $N \times N$ variables Q$_{rc}$, $r \in [1,N]$, $c \in
[1,N]$, Q$_{rc} = 1$ if cell $r,c$ in the chessboard is occupied by a
Queen, $0$ otherwise. Then the following constraints need to be
satisfied:

\begin{eqnarray*}
 Sum(Q_{rc}) & = & \left\{
   \begin{array}{l l}
     1 & \forall r \quad \mbox{(only one Queen per row)}\\
     1 & \forall c \quad \mbox{(only one Queen per column)}\\ 
   \end{array} \right. \\
Sum(Q_{r+i,c+i}) & = & 1\\
Sum(Q_{r-i,c-i}) & = & 1 \qquad \mbox{(only one Queen on each diagonal)}\\
\end{eqnarray*} 

The problem can be solved by finding assignments for all variables
Q$_{rc}$ that satisfy the above constraints. Fig. \ref{fig:nqueens-2}
is a solution found by \eu using a compact version of that formulation
and a specialized search procedure.

\begin{figure}
\centering
\includegraphics[scale=2.0]{figs/Example-NQueens1.jpeg}
\caption{\small An $N$-Queens solution generated by \eue.}
\label{fig:nqueens-2}
\end{figure}


\paragraph{} \textit{Scheduling}: The Resource Constrained Project
Scheduling Problem (\texttt{RCPSP}) \cite{Bruckera99}, consists of a
set of activities that must be scheduled in a way that satisfies
minimum and/or maximum temporal separation constraints. The activity
schedule must also respect fixed limits on the availability of
resources required to perform each activity. In addition to satisfying
temporal and resource constraints, it is common for the user to want
to minimize makespan \cite{ghallab04} so that the entire project is
finished as early as possible. Fig. \ref{fig:rcpsp-1} shows an example
of a a solution provided by \eu for an \texttt{RCPSP} instance with
$10$ activities, $5$ resources and $30$ temporal constraints.

\begin{figure}[t]
\centering
\subfloat[\small A \eu solution to an \texttt{RCPSP} \cite{Bruckera99}
  problem.]{\label{fig:rcpsp-1}\includegraphics[scale=1.8]{figs/Example-UBO0.jpeg}}\qquad
\subfloat[\small A \eu solution to a shopping agent problem domain where
  the agent needs to buy Bananas, Milk and a Drill.]{\label{fig:shopping-1}\includegraphics[scale=2.2]{figs/Example-Shopping0.jpeg}}
\caption{\small \eu based solutions to scheduling and planning problems.}
\end{figure}


\paragraph{} \textit{Planning}: The Shopping Agent Problem
\cite{russelnorvig} is a variation of the problem with which we
started this chapter.  An agent needs to purchase a set of products
(milk, drill, etc) that are available at specific locations
(supermarket, hardware store, etc). The agent is subject to temporal
(must complete tasks by specific deadlines) and resource (fuel,
carrying capacity, etc) constraints and it needs to synthesize actions
that need to be performed to find and acquire the required items, as
well as when to perform each of those actions.
Fig. \ref{fig:shopping-1} shows a solution produced by \eu for such a
problem instance.

As the examples above show, a Planning problem (actions to achieve a
goal) may embed a Scheduling problem (what resources are necessary to
achieve stated goals) and both Planning and Scheduling may embed a
Constraint Satisfaction problem.  The relationships between Planning,
Scheduling and Constraint Satisfaction have been examined
\cite{smith00} and have lead to use of constraint reasoning in \eu as
a core building block.

\subsection{Constraint Programming}
\label{sec:europa:cp}

Constraint Satisfaction Programming (also known as Constraint
Programming or \textsf{CP}) is \kcomment{an AI} discipline that
provides a generic framework for representing, solving and making
logical inference on constraints. A complete treatment of this
discipline is given in \cite{marriott98,apt03} with concise
introductions provided in \cite{bartak99,lustig01}.

A constraint programming problem consists of a set of variables $V=
{x_1,..,x_N}$, where each variable takes values from a domain
$d_1,..,d_N$; in this chapter we will deal only with discrete and
finite domains. Given a defined conjunctive set of constraints
\kcomment{\cite{gen87}} on the variables: $C=\{c_1(x_1,..,x_N), ...,
c_K(x_1,..,x_N)\}$, the objective is to find one or more value
assignments to $V$ where all constraints are satisfied. To solve a
problem, \textsf{CP} techniques use logical inference to perform
Constraint Propagation, which consists of two operations:

\begin{enumerate}

\item \textbf{Bounds propagation} inferring upper and lower variable
  bounds. For example, from the constraints $x_1 + x_2 \leq 2$ \& $x_i
  \geq 0$, we can infer $[0,2]$ bounds for $x_1$ and $x_2$.

\item \textbf{Domain reduction} inferring a valid set of values for a
  variable.  For example, for constraints
  \texttt{allDifferent}($x_1,x_2,x_3$), $x_i \geq 0$ \& $x_i \leq 4$,
  if $x_1 = 1$ and $x_2 = 3$ we can infer that the valid and reduced
  domain for $x_3$ is $\{2,4\}$.

\end{enumerate}

A set of constraints and variables that describe a problem domain are
typically represented as a \emph{Constraint Network}, where the
variables are nodes and constraints are arcs between the variables. A
pair of variables $x$ and $y$ is said to be \emph{arc consistent} if
for every value in $x$'s domain, there is a value in $y$'s domain that
is consistent with all the constraints that connect $x$ and $y$.  A
naive approach to ensuring arc consistency consists of cycling through
all the variable pairs and performing constraint propagation until
there are no variable domain changes. The \texttt{AC-3} algorithm
\cite{mackworth77}, a popular implementation, improves over the naive
approach by ignoring constraints whose variables were not affected
since the last iteration. Fig. \ref{fig:constraintnet} shows a set of
constraints on $3$ variables ($x$, $y$ and $z$) represented as a
constraint network and their domains before arc consistency is
enforced.  After the \texttt{AC-3} algorithm is executed, the domains
for the variables would be pruned to x=$\{2$\}, y=$\{2$\}, z=$\{2$\}.

\begin{figure}[!t]
\centering
\includegraphics[scale=1.9]{figs/constraint-net.jpeg}
\caption{\small Variables and Constraints represented as a Constraint
  Network \kcomment{before arc consistency is enforced}.}
\label{fig:constraintnet}
\end{figure}

Arc Consistency and other more stringent consistency definitions can
be used to efficiently prove that a \textsf{CP} problem is
\emph{unsatisfiable}. However, they generally cannot prove that a
\textsf{CP} problem is satisfiable and provide specific variable
assignments that constitute solutions. To do so, search algorithms
like global backtracking \cite{hooker05} or local search are used.
These use consistency as an efficient way to prune the search space
\cite{cp06}. In theory, solving a \textsf{CP} problem is NP-Hard
\cite{ghallab04}, but often in practice \kcomment{heuristic methods
  provide viable ways to generate efficient solutions}.

\textsf{CP} is usually implemented as part of a programming language
and constraints are usually represented as objects \cite{puget95}. Any
constraint that the user is able to reason about can be introduced
into the system and as long as the constraint propagation protocols
specified by the host \textsf{CP} system are enforced, it will be
indistinguishable from any other "primitive'' \textsf{CP} constraint,
such as $\leq$ or $\geq$.


\subsection{Constraint-Based Planning}
\label{sec:europa:cp}

The most common planning formulations use propositional
\kcomment{logic}, where the state of the world is represented by a set
of propositions (statements that can be true or false), and operators
change the truth values of these propositions \cite{gen87}. Although
these formulations are powerful \kcomment{for descriptive semantics}
in automated planning, there are classes of problem domains that are
difficult to represent. In particular, it is hard to represent time,
resources, mutual exclusion and concurrency.  Constraint Programming
is a framework that lends itself well to represent and reason about
these elements \cite{ghallab94} with the consequence that there has
been substantial effort in building automated planners that
\kcomment{leverage} \textsf{CP} techniques.  The most common approach
in constraint-based planning has been to encode the entire planning
problem as a \textsf{CP} problem, where action choices and
relationships are represented as variables and/or constraints
\cite{do01,vanbeek99,vossen99,wolfman99}.  The advantage of using a
pure \textsf{CP} formulation is that standard solvers can be then used
to solve the planning problem. However, this approach also has
significant drawbacks:

\begin{enumerate} 

\item Given that action choices and relationships (rules) are
  expressed through variables, the domain descriptions that result
  from this approach are not intuitive and therefore difficult to
  understand and debug.

\item If the structure of a planning model (actions, conditions,
  effects, dependencies at the action level) is not explicitly
  maintained by the \textsf{CP} planner, the search algorithms are
  deprived of critical information to make better decisions. If that
  structure is maintained (for instance, by internally marking
  variables that represent action choice and relationships between
  actions), it is still hard to write search algorithms as any
  planning-specific insight has to be translated into the \textsf{CP}
  representation of variables and constraints.
  
\end{enumerate}

These drawbacks have been addressed by a number of frameworks like
\texttt{Descartes} \cite{joslin95,joslin96}, \texttt{IxTet}
\cite{ghallab87,lemai04} and \texttt{CAIP} (Constraint-Based Interval
and Attribute Planning) \cite{mus94,frank2003}.  They take advantage
of a \textsf{CP} representation to deal with time and resources,
\kcomment{while} maintaining an explicit representation of the
elements relevant to planning (action choices, causal relationships,
etc). This combination results in a planning approach that can deal
with the \kcomment{kind} of constraints found in real-world problems
while allowing for intuitive problem descriptions and a natural way to
express search algorithms and heuristics. \eu is one such
implementation of the \texttt{CAIP} framework, \kcomment{others
  include \cite{chien00, iaai-CestaCFO09}}. In \eu planning problems
are specified in terms of:

\begin{description}

\item[\textbf{Intervals}:] entities that have a temporal scope (that
  is, a beginning and an end in time) and a set of
  attributes. Intervals are used to represent actions and states in
  the planning domain.

\item[\textbf{Variables and Constraints}:] Variables represent Interval
  attributes and constraints enable a direct representation of
  temporal, resource and any other restrictions that a valid plan must
  comply with.

\item[\textbf{Domain Configuration Rules}:] This is an explicit
  representation of the conceptual relationships between actions and
  states in the problem domain, for instance, for a Shopping Agent to
  be able to purchase a product, it must be at a location where the
  product is available.

\end{description}

To generate plans, the \texttt{CAIP} framework uses a plan-space
search approach \cite{ghallab04}, where a \emph{partial
  plan}\footnote{\kcomment{A partial plan is a subsequence of a plan
    which can be refined into a plan structure.}} is evolved until all
open goals are achieved and all inconsistencies are removed. The
specific search algorithm used by \eu will be covered in
\kcomment{further} detail below.


%%% Local Variables: 
%%% mode: latex
%%% TeX-master: "setobook"
%%% End: 
