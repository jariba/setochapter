\begin{table*}[b]
  \centering
  \renewcommand{\arraystretch}{1.0}%{0.99}
  \begin{tabular}{|l|l|p{7cm}|p{3cm}|}
    \hline
    \textbf{Constraint}& \textbf{Syntax}& \textbf{Description}& \textbf{Variable Types}\\
    \hline
    distanceSquares& distanceSquares(a,b,c)& c = sqrt(a+b) if a and b are singleton& Numeric intervals\\
    \hline
    calcDistance&calcDistance(a,b,c,d,e)& a is Euclidean distance between points (b,c) and (d,e)& Numeric intervals\\
    \hline
    sin& sin(a,b)& a = sin(b)& Numeric intervals\\
    \hline
    allDiff& allDiff(a,b,\ldots)& Restrict all domains so the intersection of any pair of domains is empty& Comparable\\
    \hline
    EqualMaximum& EqualMax(a,b,\ldots)& a = max(b,\ldots)& Numeric\\
    \hline
    EqualMinimum& EqualMin(a,b,\ldots)& a = min(b,\ldots)& Numeric\\
    \hline
    CountZeros& CountZeros(a,b,\ldots)& a is the count of the rest that can be zero& Numeric\\
    \hline
    CountNonZeros& CountNonZeros(a,b,\ldots)& a is the count of the rest that can be non-zero& Numeric\\
    \hline
    diffSquare& diffSquare(a,b,\ldots)& c = (a - b)$^2$ if a and b are singleton& Numeric intervals\\
    \hline
    card& card(a,b,\ldots)& a must be greater than or equal the count of the other variables that are true& Numeric\\
    \hline
  \end{tabular}
  \caption{\small Calculation Constraints: These constraints enforce that one variable (usually $a$) is constrained by a calculation done on the remaining variables.}
  \label{tab:calconst}
\end{table*}

%%%%%%%%%%%%%

\begin{table*}[ht]
  \centering
  \renewcommand{\arraystretch}{1.0}%{0.99}
  \begin{tabular}{|l|l|p{7.5cm}|p{3cm}|}
    \hline
    \textbf{Constraint}& \textbf{Syntax}& \textbf{Description}& \textbf{Variable Types}\\
    \hline
    subsetOf& subsetOf(a,b)& a is a subset of b& Comparable\\
    \hline
    memberImply& memberImply(a,b,c,d)& If a is a subset of b, then require that c is a subset of d& a and b comparable, c and d comparable\\
    \hline
    Lock& Lock(a,b)& Restrict a's domain to be contained in b's domain& Comparable\\
    \hline
  \end{tabular}
  \caption{\small Set Constraints: These constraints are likely to be used for set comparisons etc. Note, however, that many of the other constraints could be used for sets (\ie Comparable variables) just as these constraints could be applied to numeric domains as well.}
  \label{tab:setconst}
\end{table*}

%%%%%%%%%%%%%

\begin{table*}[ht]
  \centering
  \renewcommand{\arraystretch}{1.0}%{0.99}
  \begin{tabular}{|l|l|p{7cm}|p{3cm}|}
    \hline
    \textbf{Constraint}& \textbf{Syntax}& \textbf{Description}& \textbf{Variable Types}\\
    \hline
    commonAncestor& commonAncestor(a,b,c)& a and b must be contained by the same object in c (or by c itself)& a, b and c are all objects\\
    \hline
    hasAncestor& hasAncestor(a,b)& a must be contained by some object in b& a and b are objects\\
    \hline
  \end{tabular}
  \caption{\small Object Hierarchy Constraints: These constraints are imposed on objects and tokens. These constraints are used to assert which object one or more tokens is contained by. Most often, the commonAncestor constraint is used to subgoal across timelines which must share a contained object in common.}
  \label{tab:objhierarchy}
\end{table*}

%%%%%%%%%%%%%

\begin{table*}[ht]
  \centering
  \renewcommand{\arraystretch}{1.0}%{0.99}
  \begin{tabular}{|l|l|p{9.5cm}|p{3cm}|}
    \hline
    \textbf{Constraint}& \textbf{Syntax}& \textbf{Description}& \textbf{Variable Types}\\
    \hline
    UNARY& NA& Restrict a variable's domain; given a variable and a domain, intersect them. Note that this constraints is only available internally and not exposed in NDDL& Anything\\
    \hline
    or& or(a,b\ldots)& At least one of the variables must be true& Numeric\\
    \hline
    absVal& absVal(a,b)& a.lb $\geq$ 0, a.ub = max(abs(b.lb)), b.lb $\geq$ -a.lb, b.ub $\leq$ a.ub& Numeric\\
    \hline
  \end{tabular}
 \caption{\small Miscellaneous Constraints}
  \label{tab:misc}
\end{table*}

